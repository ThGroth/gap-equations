\documentclass[a4paper,11pt]{amsart}
\usepackage{geometry}
\geometry{a4paper,left=35mm,right=35mm, top = 35mm, bottom=35mm}

\usepackage[utf8]{inputenc}
\usepackage[T1]{fontenc}
\usepackage{setspace}
%\usepackage[ngerman]{babel}
\usepackage{lmodern}


\usepackage{amsmath}
\usepackage{amsthm}
\usepackage{amssymb}
% Boldface Zahlen
\usepackage{bbm}
\usepackage{enumerate}
\usepackage{mathtools}

%For include without pagebreak
\usepackage{newclude}

% For relative Textsizes
\usepackage{relsize}

%For units
\usepackage[binary-units=true]{siunitx}

\usepackage[disable]{todonotes}

\usepackage{graphicx}
% Bibliographi
%\usepackage{natbib}

%Für Quellcode
\usepackage{listings}

%Für Verlinkungen im Dokument
\usepackage{xcolor}
\definecolor{white}{rgb}{1,1,1}
\usepackage[linkbordercolor=white,urlbordercolor=white ,citebordercolor=white, plainpages=false]{hyperref}

%Zum Erstellen von Graphen für Automaten
\usepackage{tikz}
\usetikzlibrary{arrows,shapes.geometric,automata,positioning}
\tikzset{elliptic state/.style={draw,ellipse}}

%Diagrams
\usepackage[arrow, matrix, curve]{xy}


%For extra small fraction \sfrac 
\usepackage{xfrac}

%Reset the equation counter after each subsection
\usepackage{chngcntr}
\counterwithin*{equation}{section}
\counterwithin*{equation}{subsection}
%Für Zeilennummern
%\usepackage[modulo]{lineno}
%\linenumbers

%Counter für Automaten
\newcounter{automatonnumber}[section]
\renewcommand{\theautomatonnumber}{\mathcal{A}_{\thesection,\arabic{automatonnumber}}}
\newcommand{\autno}[1]{
  \refstepcounter{automatonnumber}
  \label{#1}
  (\theautomatonnumber)
}

%Quellcode Aussehen:
\lstset {
basicstyle=\normalsize,
language=GAP,
breaklines=true,
showstringspaces=false,
tabsize=2
}
%Monospace in inline code
%\lstMakeShortInline[columns=fixed]|

\newtheorem{pro}{Proposition}[section]
\newtheorem{con}[pro]{Conjecture}
\newtheorem{thm}[pro]{Theorem}
\newtheorem{cor}[pro]{Corollary}
\newtheorem{lem}[pro]{Lemma}

\newtheorem{task}{Task}

\theoremstyle{definition}
\newtheorem*{ex}{Example}
\newtheorem*{re}{Remark}
\newtheorem{defi}[pro]{Definition}
\newtheorem{cordef}[pro]{Corolary/Definition}


%Für Quotienten
\newcommand{\RRNK}[2]{
  \raisebox{1ex}{\ensuremath{#1}}
  \ensuremath{\mkern-3mu} \bigg/  \ensuremath{\mkern-3mu}
  \raisebox{-1ex}{\ensuremath{#2}}
}
\newcommand{\RNK}[2]{{#1/#2}}
% \newcommand{\RNK}[2]{
%   \mathchoice{ \raisebox{1ex}{\ensuremath{#1}}%
% 	       \ensuremath{\mkern-4mu} \big/  \ensuremath{\mkern-4mu}%
% 	       \raisebox{-1ex}{\ensuremath{#2}}}%
% 	     { \raisebox{0.2ex}{\ensuremath{#1}}%
% 	       \ensuremath{\mkern-4mu} /  \ensuremath{\mkern-4mu}%
% 	       \raisebox{-0.2ex}{\ensuremath{#2}}}%
% 	     { \raisebox{0.1ex}{\ensuremath{\scriptstyle#1}}%
% 	       \ensuremath{\mkern-4mu} /  \ensuremath{\mkern-4mu}%
% 	       \raisebox{-0.1ex}{\ensuremath{\scriptstyle#2}}}%
% 	     { {\ensuremath{#1}}%
% 	        /%
% 	       {\ensuremath{#2}}}%
%   \ 
% }

%uppergausian bracket
\DeclarePairedDelimiter{\ceil}{\lceil}{\rceil}

%Pfeile für exakte Sequenzen
\newcommand{\exar}[1][ ]{\overset{#1}{\longrightarrow}}
%Normalteiler/Ideal 
\newcommand{\normal}{{\mathrel\trianglelefteq}}

%großer Strich
\newcommand{\bigmid}{\ \middle \vert }

%Besondere Buchstaben
\newcommand{\NN}{\mathbb{N}}
\newcommand{\ZZ}{\mathbb{Z}}
\newcommand{\QQ}{\mathbb{Q}}
\newcommand{\RR}{\mathbb{R}}
\newcommand{\CC}{\mathbb{C}}
\newcommand{\HH}{\mathbb{H}}
\newcommand{\PP}{\mathbb{P}}
\newcommand{\one}{\mathbbm{1}}

%das d hinter dem Integral
\newcommand{\intd}{d}

%Sprachen
\newcommand{\La}{\mathcal{L}}
\newcommand{\emptyword}{\epsilon}
%Zustandmenge
\newcommand{\States}{\zeta}
%Alphabet
\newcommand{\Al}{\mathcal{A}}
%Zentrum
\newcommand{\Ce}{\mathfrak{C}}
%Free Variables in a groupword
\newcommand{\Var}{\textup{Var}}
%Commutator width
\newcommand{\cw}{\textup{width}}
%Stabilizer
\newcommand{\Sta}{\textup{St}}
%Projektion auf Alphabet
\newcommand{\alphab}{\mathfrak{wr}}
\newcommand{\state}{\mathfrak{state}} 
%zugehörige Mealy
\newcommand{\mealy}[1]{\mathfrak{M}_{#1}}
%Aktivität eines Automorphismus
%\newcommand{\act}{\mathcal{A}\mathcal{CT}}
%\newcommand{\act}{{\scalebox{0.9}{$\mathcal{A}$}\mspace{-2mu}\raisebox{-0.3ex}{\tiny$\mathbf{ct}$}}}
\newcommand{\act}{{\textup{\smaller$\mathcal{A}\mathbf{ct}$}}}
%PairKlammern
\DeclarePairedDelimiter{\pair}{\langle\negthickspace\langle}{\rangle\negthickspace\rangle}
%\newcommand{\pair}[1]{{\langle\negthickspace\langle #1 \rangle\negthickspace\rangle}}
%Nucleus
\newcommand{\Nuc}{\mathcal{N}}
%Commutatror Width
\newcommand{\mcount}{{\#_M}}
%Normal normal_form
\newcommand{\nf}{\mathfrak{nf}}
%Reduced constraints
\newcommand{\Red}{\mathfrak{R}}
%Ideale
\newcommand{\ai}{\mathfrak{a}}
\newcommand{\bi}{\mathfrak{b}}
\newcommand{\Ji}{\mathcal{Ji}}

%Zustände von Automaten

%\newcommand{\at}[1]{  {\raisebox{-0.3ex}[1ex]{$\mathbin{@}$}#1} }
\newcommand{\at}[1]{  {{\textup{\smaller @}}#1} }
\newcommand{\att}[1]{  {{\bar{\textup{\smaller @}}}#1} }
%\newcommand{\at}[1]{  |_{#1} }

%Centralizer:
\newcommand{\centra}{\mathcal{C}}
%Stabilizer
\newcommand{\Stab}{\textup{Stab}}
%Undefined elememts
\newcommand{\undef}{\varnothing}

%Funktionen aufrecht
\newcommand{\Aut}{\textup{Aut}}
\newcommand{\RAut}{\textup{RAut}}
\newcommand{\FAut}{\textup{FAut}}
\newcommand{\Poly}{\textup{Poly}}
\newcommand{\EPoly}{\mbox{\smaller$\frac{1}{2}$}\Poly}
%\newcommand{\EPoly}{\textup{EPoly}}
\newcommand{\Orb}{\textup{Orb}}
\newcommand{\OS}{\textup{OS}}
\newcommand{\OSA}{\textup{OSA}}
\newcommand{\supp}{\textup{supp}}
\newcommand{\End}{\textup{End}}
\newcommand{\rad}{\textup{rad}}
\newcommand{\ad}{\textup{ad}}
\newcommand{\id}{\mathbbm{1}}
\newcommand{\im}{\textup{i}}
\newcommand{\Min}{\textup{Min}}
\newcommand{\Imm}{\textup{Im}}
\newcommand{\Span}{\textup{span}}
\newcommand{\Spec}{\textup{Spec}}
\newcommand{\Spur}{\textup{tr}}
\newcommand{\rk}{\textup{rk}}
\newcommand{\sign}{\textup{sign}}
\newcommand{\diag}{\textup{diag}}
\newcommand{\Sym}{\textup{Sym}}
\newcommand{\ord}{\textup{ord}}
\newcommand{\suc}{\textup{suc}}
\DeclareMathOperator{\lcm}{lcm}
\DeclareMathOperator{\operp}{{ \bigcirc\!\!\!\!\!\!\!\perp}}

\newcommand{\BIGOP}[1]{\mathop{\mathchoice%
{\raise-0.22em\hbox{\huge $#1$}}%
{\raise-0.05em\hbox{\Large $#1$}}{\hbox{\large $#1$}}{#1}}}
\newcommand{\bigtimes}{\BIGOP{\times}}

\newcommand{\bigcomp}{\BIGOP{\circ}}

% nur fuer Bigboxplus andere Korrekturen
\newcommand{\BIGboxplus}{\mathop{\mathchoice%
{\raise-0.35em\hbox{\huge $\boxplus$}}%
{\raise-0.15em\hbox{\Large $\boxplus$}}{\hbox{\large $\boxplus$}}{\boxplus}}}
\parindent=0em

% For filenames
\newcommand{\filename}[1]{{\texttt{#1}}}
%TODO replace this by a nicer one
\newcommand{\gapinline}[1]{{\lstinline{#1}}}
%\bibliographystyle{THdeutsch}
%\bibliographystyle{alphaTH}
%\bibliographystyle{dinat}
%\bibliographystyle{TH}
\bibliographystyle{latex/hamsalpha}

\begin{document}
\title{Commutator width in the first Grigorchuk group}
\author{Laurent Bartholdi}
\author{Thorsten Groth}
\author{Igor Lysionok}
\begin{abstract}
Let $G$ be the Grigorchuk group. In \cite{Lysenok:QudraticEquationsInGrig} it was shown that the commutator width of $G$ is finite but no explicit bound was given.
In the present paper we show that in fact each element of the derived subgroup $g\in G'$ is a product of two commutators. This means that all equations
of the form $[x_1,x_2][x_3,x_4]g=1$ are solvable for $g\in G'$. The computer algebra system \cite{GAP4} is used to derive a series of equations with increasing 
genus. 
\end{abstract}
\maketitle
\tableofcontents

%%%%%%%%%%%%%%%%%%%%%%%%%%%%%%%%%%%%%%%%%%%%%%%%%%%%%%%%%%%%%%%%
\section{Introduction}
%TODO
INTRO: groups acting on trees. Decision problems. Importance of
quadratic equations. Commutator width, simplest example of $2$-group
with width $\ge2$, show that it appears in Grig. gp.

\begin{thm}
 The Grigorchuk group $G$ has commutator width 2.
\end{thm}
ALSO: all finite index subgroups have c.w. finite; maybe even all 2?

ALSO: say this is a new proof, that does not require LMU

ALSO: say that the proof is GAP-assisted; verification of the arguments is done by running GAP code, and checking the proof === checking the GAP code.

ALSO: quote Elisabeth Fink~\cite[Question 3]{Fink:Conjugacy_growth} in how this gives bounds on the conjugacy width of $G$: every element is a product of at most $4$ conjugates of $a$, e.g.

%%%%%%%%%%%%%%%%%%%%%%%%%%%%%%%%%%%%%%%%%%%%%%%%%%%%%%%%%%%%%%%%
\section{Equations}
In this section some standard notations similar to the ones introduced
in~\cite{ComerfordEquationsFreeGroups} are established. We fix a set
$X$ of \emph{variables}. As it should always be infinite countable it
can be assumed to be in bijection with $\NN$. Implicitly, it is well
ordered, and its family of finite subsets is also well ordered, by
size and then lexicographic order. We denote by $F_X$ the free group
on the generating set $X$.

\begin{defi}
  Let $G$ be a group. A \emph{$G$-group} is a group with a
  distinguished copy of $G$ inside it; e.g.\ $G*H$ for some group
  $H$. A \emph{$G$-homomorphism} between $G$-groups is a homomorphism
  that is the identity between the marked copies of $G$.

  A \emph{$G$-equation} is an element $E$ of the group $F_X * G$,
  regarded as reduced word. For $E$ a $G$-equation, define its set of
  \emph{variables} $\Var(E)\subset X$ as the set of symbols in $X$
  that occur in it; namely, $\Var(E)$ is the minimal subset of $X$
  such that $E$ belongs to $F_{\Var(E)}$.

  An \emph{evaluation} is a $G$-endomorphism $e\colon F_X * G \to G$.
  A \emph{solution} of an equation $E$ is an evaluation $s$ with
  $s(E)=1$. If a solution exists for $E$ then the equation $E$ is
  called \emph{solvable}, and one writes $\models E$.  The set of
  elements $x\in X$ with $s(x)\neq 1$ is called the \emph{support} of
  the solution.
\end{defi}

Often the support of a solution for an equation $E$ is assumed to be
minimal and thus a subset of $F_{\Var(E)}$.  As the solution is
uniquely described by the image of $X$ the data of a minimal solution
is equivalent to a map $\Var(E) \to G$.  The question of whether an
equation $E$ is solvable will be referred to as the \emph{diophantine}
problem of $E$.

Every homomorphism $\varphi \colon G \to H$ extends uniquely to an
$F_X$-homomorphism $\varphi_* \colon F_X*G \to F_X*H$. In this manner,
every $G$-equation $E$ gives rise to an $H$-equation $\varphi_*(E)$,
which is solvable whenever $E$ is solvable.

\begin{defi}
  Let $E,F\in F_X* G$ be two equations. We say $E$ and $F$ are
  \emph{equivalent} if there is a $G$-automorphism $\varphi$ of
  $F_X*G$ which maps $E$ to $F$.
\end{defi}
\begin{lem}
  Let $E$ be an equation and let $\varphi$ be a $G$-endomorphism of
  $F_X*G$. If $\varphi(E)$ is solvable then so is $E$. In particular,
  the diophantine problem is the same for equivalent equations.
\end{lem}
\begin{proof}
  If $s$ is a solution for $\varphi(E)$, then $s\circ\varphi$ is a
  solution for $E$.
\end{proof}

\subsection{Quadratic equations}
A $G$-equation $E$ is called \emph{quadratic} if for each variable
$x\in \Var(E)$ exactly two letters of $E$ are $x$ or $x^{-1}$, when
$E$ is regarded as a reduced word.

A $G$-equation $E$ is is called \emph{oriented} if for each variable
$x\in \Var(E)$ the number of occurrences with positive and with
negative sign coincide, namely if $E$ maps to the identity under the
natural map $F_X*G\to F_X/[F_X,F_X]*1$.  Otherwise $E$ is called
\emph{unoriented}.
\begin{lem}
 Being oriented or not is the same for equivalent equations.
\end{lem}
\begin{proof}
  $E$ is oriented if and only if it belongs to the normal closure of
  $[F_X,F_X]*G$; this subgroup is preserved by all $G$-endomorphisms
  of $F_X*G$.
\end{proof}

\subsection{Normal form of quadratic equations} \label{sec:normal_form}
\begin{defi}
  For $x_i,y_i,z_i \in X$ and $c_i \in G$ the following two kinds of
  equations are called in \emph{normal form}:
 \begin{align}
  O_{n,m}:\qquad & [x_1,y_1][x_2,y_2]\cdots[x_n,y_n]c_1^{z_1}\cdots c_{m-1}^{z_{m-1}}c_m  \\
   U_{n,m}:\qquad & x_1^2x_2^2\cdots x_n^2 c_1^{z_1}\cdots c_{m-1}^{z_{m-1}}c_m\ .
 \end{align} 
 The form $O_{n,m}$ is called the oriented case and $U_{n,m}$ for
 $n>0$ the unoriented.  The parameter $n$ is referred to as
 \emph{genus} of the normal form of an equation.
\end{defi}

We shall prove the following theorem:
\begin{thm}[\cite{ComerfordEquationsFreeGroups}] \label{Thm:equationNormalForm}
  Every quadratic equation $E \in F_X*G$ is equivalent to an equation
  in normal form, and the isomorphism can be effectively computed.
\end{thm}

\begin{proof}
  The proof goes through an induction on the number of variables.
  Starting with the oriented case: If the reduced equation $E$ has no
  variables then it is already in normal form $O_{0,1}$. If there is a
  variable $x\in X$ occurring in $E$ then it also appears with
  opposite sign.  Therefore, the equation has the form
  $E = ux^{-1}vxw$ or can be brought to this form by applying the
  automorphism $x \mapsto x^{-1}$. Choose $x\in X$ in such a way that
  $\Var(v)$ is minimal.
 
  We distinguish between multiple cases:
  \begin{itemize}
  \item[Case $1.0$] $v\in G$. The word $uw$ has fewer variables than
    $E$ and can thus be brought into normal form $N\in O_{r,s}$ by
    $G$-isomorphism $\varphi$. If $N$ ends with a variable we can use
    the $G$-isomorphism $\varphi \circ (x\mapsto xw^{-1}) $ to map $E$
    to the equation $Nv^x \in O_{r,s+1}$.
    
    If $N$ ends with a group constant $b$, $N=Mb$ we can use the
    isomorphism $\varphi \circ(x \mapsto xbw^{-1}) $ to map $E$ to the
    equation $Mv^xb\in O_{r,s+1}$.

  \item[Case $1.1$] $v\in X\cup X^{-1}$. For simplicity let us assume
    that $v\in X$. In the other case we can apply $v \mapsto v^{-1}$.
    Now there are two possibilities: either $v^{-1} \in u$ or
    $v^{-1} \in w$. In the first case $E= u_1v^{-1}u_2x^{-1}vxw$, and
    then the isomorphism $x \mapsto x^{u_1}u_2$, $v \mapsto v^{u_1}$
    results in the equation $[v,x]u_1u_2w$. In the second case
    $E= ux^{-1}vxw_1v^{-1}w_2$ is transformed to $[x,v]uw_1w_2$ by the
    isomorphism $x \mapsto x^{uw_1}w_1^{-1}$, $v\mapsto v^{-uw_1}$. In
    both cases $u_1u_2w$, respectively $uw_1w_2$ have fewer variables
    and so composition with the corresponding isomorphism results in a
    normal form.
  \item[Case $2$] Length$(v)>1$. Then $v$ is a word consisting of
    elements $X\cup X^{-1}$ with each symbol occurring at most once as
    $v$ was chosen with minimal variable set, and some elements of
    $G$.  If $v$ starts with a constant $b\in G$ we can use the
    homomorphism $x\mapsto bx$ to achieve that $v$ starts with a
    variable $y\in X$, possibly using $y \mapsto y^{-1}$. As in Case
    $1.1$ there are two possibilities: $y^{-1}$ is either part of $u$
    or part of $w$. In the first case $E= u_1 y^{-1} u_2 x^{-1}yv_1xw$
    we can use the isomorphism $x\mapsto x^{u_1v_1}u_2$,
    $y\mapsto y^{u_1v_1}v_1^{-1}$ to obtain $[y,x]u_1v_1u_2w$. In the
    second we use the isomorphism
    $x\mapsto x^{uw_1v_1}v_1^{-1}w_1^{-1}$,
    $y\mapsto y^{-uw_1v_1}v_1^{-1}$ to obtain $[x,y]uw_1v_1w_2$. In
    both cases the second subword has again less variable and can be
    brought into normal form by induction.
  \end{itemize}
  Therefore each oriented equation can be brought to normal form by a
  $G$-isomorphisms.

  In the unoriented case decompose the equation into $E = uxvxw$ with
  again $v$ having a minimal number of variables.  The shorter word
  $uv^{-1}w$ is equivalent by $\varphi$ to a normal form $N$ by
  induction.
 
  The $G$-isomorphism $\varphi \circ (x\mapsto x^uv^{-1})$ maps $E$ to
  $x^2N$. If $N\in U_{r,s}\cup\thinspace O_{0,t}$ for some $r,s,t$,
  there remains nothing to do.  Otherwise $N=[y,z]M$, and then the
  homomorphism
  \begin{align*}
    x&\mapsto xyz, & y&\mapsto z^{-1}y^{-1}x^{-1}yzxyz, & z&\mapsto z^{-1}y^{-1}x^{-1}z
  \end{align*}
  maps $x^2N$ to $x^2y^2z^2M$. This homomorphism is indeed an
  isomorphism, with inverse
  \begin{align*}
    x&\mapsto x^2y^{-1}x^{-1}, & y&\mapsto xyx^{-1}z^{-1}x^{-1}, & z&\mapsto xz.
  \end{align*}
  If $M$ is still not in $O_{0,s}$ this procedure can be repeated with
  $z$ instead of $x$.
\end{proof}
For a quadratic equation $E$ we denote by $\Normal(E) := \Normal_E(E)$
the image of $E$ under the $G$-isomorphism $\Normal_E$ constructed in
the proof.

From now on we will consider oriented equations $O_{(n,1)}$. For this
we will use the abbreviation
\[R_n(x_1,\ldots,x_{2n})=\prod_{i=1}^n [x_{2i-1},x_{2i}]\]
and often write $R_n=R_n(x_1,\ldots,x_{2n})$ if the $x_i$ are the
first generators of $F_X$.

\subsection{Constrained equations}
\begin{defi}[\cite{Lysenok:QudraticEquationsInGrig}]
  Given an equation $E \in F_X*G$, a group $H$, a homomorphism
  $\pi\colon G \to H$ and a homomorphism $\gamma\colon F_X \to H$, the
  pair $(E,\gamma)$ is called a \emph{constrained} equation and
  $\gamma$ is called a \emph{constraint} for the equation $E$ on $H$.
 
  A solution for $(E,\gamma)$ is a solution $s$ for $E$ with the
  additional property that $s\circ\pi=\gamma$.
\end{defi}

\section{The first Grigorchuk Group}
Let $T_n$ be an infinite regular rooted $n$-ary tree and $S_n$ the symmetric group on $n$ symbols.
The group $\Aut(T_n)$ consists of all root preserving graph automorphisms of the tree $T_n$. 
Note that $T_n$ is isomorphic to any $n$-ary subtree and therefore there is an isomorphism
$\psi\colon \Aut(T_n)\stackrel{\simeq}{\to} \Aut(T)\wr S_n$. 
%TODO!!!LB: Stopped checking the grammar!!!


A self similar subgroup of $\Aut(T_n)$ is a group $G$ with $G< \psi(G)$ For the sake of an easy notation we
will identify elements with the image of these embedding and will write $g=\pair{g_1,\ldots,g_n}\pi$ for elements $g\in G$. 
Furthermore we will call the $g_i$ states of the element $g$ and write $g\at{i}:=g_i$. 

The Grigorchuk $2$-group is a finitely generated self-similar group with finite state generators:
\[a=\pair{\id,\id}(1,2),\quad b=\pair{a,c},\quad c=\pair{a,d},\quad d=\pair{\id,b}\ . \]

Some useful identities are
\begin{itemize}
 \item $a^2=b^2=c^2=d^2=\id$ 
 \item $b^a= \pair{c,a}, c^a=\pair{d,a}, d^a=\pair{b,\id}$
 \item $(ad)^4=(ac)^8=(ab)^{16}= \id$.
\end{itemize}

\begin{lem}
The Grigorchuk group is regular branched with branching subgroup 
 $K:= \left< (ab)^2,(bada)^2,(abad)^2 \right>$.  
 
 The Quotient $Q := \RNK{G}{K}$ is of order $16$.
\end{lem}

\begin{lem}[\cite{Lysenok:QudraticEquationsInGrig}]
 Given $n\in \NN$ and any homomorphism $\gamma\colon F_X \to Q$ with $\supp(\gamma) \subset \left<x_1,\ldots x_{2n}\right>$
 there is an element $\varphi \in \Stab(R_n)<\Aut(F_X)$ such that $\supp(\gamma\circ\varphi) \in \left<x_1,\ldots,x_5\right>$.
\end{lem}
\begin{re}
 This is now implemented in GAP: For this see the 
 attached gap file and the function \emph{ReduceConstraint} which can be called 
 with one argument being either a group homomorphism from a free group to $Q$ or 
 a list of elements of $Q$. The group $Q$ can be accessed by
 \lstinline{BranchStructure(GrigorchukGroup).group} which comes with the GAP package FR \cite{FR2.1.1}.
\end{re}

\begin{lem}
 Identify the group of homomorphisms $\{\gamma \colon F_X \to Q \mid \supp(\gamma) \subset \left<x_1,\ldots x_6\right>\}$ with $Q^6$. 
 Then \[\left|\RNK{Q^6}{\Stab(R_3)}\right|=90.\]
\end{lem}
\begin{proof}
%TODO Reference for the generators of the mcg. 
 This is shown by a GAP calculation. The group $\Stab(R_n)$ is the mapping class group
 of the surface group of the oriented surface of genus $n$ and can be generated by the 
 following automorphisms of $F_{2n}$:
 \begin{align*}
  \varphi_i&\colon& x_i&\mapsto x_{i-1}x_i & \textup{for }&i=2,4,\ldots,2n \\
  \varphi_i&\colon&x_i&\mapsto x_{i+1}x_i & \textup{for }&i=1,3,\ldots,2n-1 \\
  \psi_i&\colon&x_i&\mapsto x_{i+1}x_{i+2}^{-1}x_i, && \\  
  &&x_{i+1}&\mapsto x_{i+1}x_{i+2}^{-1}x_{i+1}x_{i+2}x_{i+1}^{-1}, && \\  
  &&x_{i+2}&\mapsto x_{i+1}x_{i+2}^{-1}x_{i+2}x_{i+2}x_{i+1}^{-1}, && \\  
  &&x_{i+3}&\mapsto x_{i+1}x_{i+2}^{-1}x_{i+3}& \textup{for }&i=1,3,\ldots,2n-3   
 \end{align*}
 Now the GAP included standard orbit enumeration algorithm \lstinline{OrbitsDomain} can be used to 
 to compute the orbit and verify its size. 
 
 To check this the function \lstinline{verifyLemma66orbits} from the file \emph{verify.g} 
 can be used. 
%  With argument \lstinline{true} it uses the precomputed orbits and just checks
%  their length. Otherwise it recomputes the orbits which can take up to $\SI{24}{\hour}$ 
%  and about $\SI{7.5}{\giga\byte}$.
%  of ram. 
\end{proof}
\begin{defi}
 Fix some representative system $\Red$ of the above $90$ orbits and for 
 $\gamma \colon F_X \to Q$ with finite support denote by $\varphi_\gamma$ the $G$-homomorphism in $\Stab(R_\NN)$ such that $\gamma\circ \varphi_\gamma \in \Red$.
 
 The element $\gamma \circ \varphi_\gamma$ will be called reduced constraint.
\end{defi}

\begin{lem} \label{lem:solvabilityWithReducedConstraint}
 The solvability of a constrained equation $(R_n g,\gamma)$ is equivalent to the solvability of $(R_n g,\gamma\circ \varphi_\gamma)$.
\end{lem}
 \begin{proof}
 If $s$ is a solution for $(R_n g,\gamma)$ then $s\circ \varphi_\gamma$ is
 a solution for $(R_n g,\gamma\circ \varphi_\gamma)$. 
\end{proof}


\begin{defi}[\cite{Bartholdi:RepresentationZetaFunctions}] 
A \emph{branch structure} of a group $G\hookrightarrow G \wr P$ consists of  
\begin{itemize}
 \item a branching subgroup $K\normal G$ of finite index. 
 \item the corresponding quotient $Q=\RNK{G}{K}$ and the factor homomorphism $\pi\colon G \to Q$.
 %TODO Q_1 or \mathcal{F}
 \item A group $Q_1 \subset Q \wr P$ such that $\pair{q_1,q_2}\sigma\in Q_1$ if and only if $\pair{g_1,g_2}\sigma \in G$ for all $g_i \in \pi^{-1}(q_i)$.
 \item A map $\omega\colon Q_1 \to Q$ with the following property. If $g=\pair{g_1,g_2}\sigma \in G$ then $\omega(\pair{\pi(g_1),\pi(g_2)}\sigma) = \pi(g)$.
\end{itemize}
\end{defi}
\begin{lem}
 The Grigorchuk group has a branch structure.
\end{lem}
\begin{proof}
 This can be seen in \cite{Bartholdi:RepresentationZetaFunctions}. In GAP the branch structure can be computed
 by the method \lstinline{BranchStructure(GrigorchukGroup)} which is included in the FR package. 
\end{proof}


\begin{thm}[\cite{Lysenok:QudraticEquationsInGrig}]\label{IgorsThm}
 The Grigorchuk group has finite commutator width. That is
 there exists an $N\in\NN$ such that for all $g\in G'$ the equation $R_Ng$ is solvable.
\end{thm}
\begin{re}
 This is not true for all constrained equations: For example
 \[\left(R_n(ab)^2,(\gamma\colon x_i\mapsto 1\ \forall i)\right)\] 
 is not solvable for any $n$ because otherwise it would be $ac,ca\in G'$ 
 which is impossible since $ac$ has activity.
\end{re}

\subsection{Good Pairs}
The previous remark motivates the following definition.
\begin{defi}
Given $g\in G'$ and $\gamma\in\Red$ . The tuple $(g,\gamma)$ is called a \emph{good pair} if there exists an $n$ such that
$(R_ng,\gamma)$ is solvable.  
\end{defi}

\begin{lem}
 Denote by 
 \[\tau\colon G\to\RNK{G}{K'}\quad\text{ and }\quad\varpi\colon \RNK{G}{K'} \to \RNK{\RNK{G}{K'}}{\RNK{K}{K'}}\simeq \RNK{G}{K}\]
 the natural projections.
 
 The pair $(g,\gamma)$ is a good pair if and only if there is a solution $s\colon F_X \to \RNK{G}{K'}$ for $R_3g^\tau$ with $s(x_i) \in \varpi^{-1}(\gamma(x_i))$. 
\end{lem}
\begin{proof}
 If $(g,\gamma)$ is a good pair and $s$ a solution for $R_ng,\gamma$ 
 then $s(x_i)\in K$ for $i\geq6$, so $s(R_n) = s(R_3) \cdot k'$ for some $k'\in K'$. 
 Therefore there is a solution $\tau\circ s$ for $R_3g^\tau$ with $s(x_i) = \gamma(x_i)$.
 
 On the other hand if there is a solution $s\colon F_X \to \RNK{G}{K'}$ for $R_3g^\tau$ with for each $s(x_i) \in \varpi^{-1}(\gamma(x_i))$ then
 for $g_i \in \tau^{-1}(s(x_i))$ there is some $k'\in K'$ such that $R_n(x_1,\ldots,x_6)gk'=1$ and so $(g,\gamma)$ is a good pair.
\end{proof}
The previous lemma shows that the question if $(g,\gamma)$ is a good pair depends only on the image $q=g^\tau$ in $\RNK{G}{K'}$. So 
$(q,\gamma)$ will be called a good pair if $(g,\gamma)$ is a good pair for one (and hence all) preimages of $q$ under $\tau$.
\begin{cor}\label{cor:finiteCommutatorWidthKimpliesBoundedConstraintedCommutators}
The following are equivalent:
\begin{enumerate}[a)]
 \item $K$ is of finite commutator width. \label{Cor:EqStatement1}
 \item There is a $N\in\NN$ uniform for all good pairs $(g,\gamma),g \in G',\gamma\in\Red$ such that $(R_Ng,\gamma)$ is solvable.
\end{enumerate}
\end{cor}
\begin{proof}
 First the easy direction: If $k\in K'$ then $(k,\id)$ is a good pair. So $(R_nk,\id)$ is solvable in $G$ for an $n\leq N$ but the constraints ensures that it is solvable in $K$.
 Therefore the commutator width of $K$ is at most $N$.
 
 If $(g,\gamma)$ is a good pair there is an $m\in \NN$ and a solution $s$ for $R_mg,\gamma$. As $s(x_i)^\pi =1$ for all $i\geq 6$ there is some $k\in K'$ such that $s$ is 
 a solution for $R_3kg,\gamma$. By \ref{Cor:EqStatement1}) there is an $N$ such that all $k\in K'$ can be written as product of $N$ commutators of elements of $K$ and
 therefore there is a solution for $(R_{N+3}g,\gamma)$.
\end{proof}
This motivates to study $K'$ and $\RNK{G}{K'}$ further.
\begin{lem} \label{lem:subgroupsOfG}
Denote by $k_1 := (ab)^2, k_2:=\pair{1,k_1} = (abad)^2$ and $k_3:=\pair{k_1,1} = (bada)^2$ then 
 \begin{align*}
  G' &= \left< k_1,k_2,k_3,(ad)^2\right>, \\
  K &= \left< k_1,k_2,k_3\right>, \\
  K\times K &= \{\pair{k,k'} \mid k,k'\in K\} \\&= \left< k_2,k_3,k_2k_1^{-1}k_2^{-1}k_1,(k_2k_1^{-1}k_2^{-1}k_1)^a,k_2k_1k_2k_1^{-1},(k_2k_1k_2k_1^{-1})^a  \right>,\\
  K' &= \left< [k_1,k_2] \right>^G \\ 
  &= \left<(dacabaca)^2(baca)^4,((ca)^2baca)^2,(dacabaca)^2c(acab)^3acad,\right.\\
  &\qquad\left.((ac)^3ab)^2, bacadacab(ac)^2(acab)^3, (acadacab)^2(acab)^4\right>^{1,a}\ .
 \end{align*}
Furthermore we have this chain of indices:
\begin{align*}
  [G:G']&=8, & [G':K]&=2, &[K:K\times K]&= 4, &[K\times K:K']&=16. 
\end{align*}
\end{lem}
\begin{proof}
 This is shown in \cite{Bartholdi:BranchGroups} and can be verified using the GAP standard methods
 \lstinline{NormalClosure} and \lstinline{Index}. 
\end{proof}

\subsection{Succesing pairs}
\begin{defi}
 We define the activity of an element $q\in Q$ as the activity of an arbitrary element of $\pi^{-1}(q)$. 
 This is well defined as $K<\Stab(1)$. 
%TODO need to define Stab(1)?
 
 Consider a constraint $\gamma\colon F_X \to Q$. 
 Define $\act(\gamma):= x \mapsto \act(\gamma(x))$.
 
 Denote by $\Red_{act}$ the reduced constraints which have a nontrivial activity.
\end{defi}

\begin{lem} \label{lem:existsGoodGamma}
 For each $q\in\RNK{G'}{K'}$ there is $\gamma\in \Red_{act}$ such that $(q,\gamma)$ is a 
 good pair.
\end{lem}
\begin{proof}
 This is a finite problem which can be checked in GAP with the function \lstinline{verifyLemmaExistGoodGammas}.
%  If this function is called with the argument \lstinline{true} then it uses the pre-computed list
%  of good pairs. Otherwise it computes the list of good pairs first, which will take about $\SI{1}{\min} $.
\end{proof}

Denote by $S$ the set $\{1,a,b,c,d,ab,ad,ba\}\subset G$.
We will define a map $\Gamma^q$ which maps a constraint $\gamma\colon F_{2n} \to Q$ with nontrivial activity to a a set of constraints $\gamma'\colon F_{4n-1}\to Q$ 
with the following properties:
\begin{itemize}
  \item There is a generator $y_{\gamma'}$ in $F_{4n-1}$ and $x\in S$ such that $\gamma'(y_{\gamma'})=x^\pi$. Denote by $F_{4n-2}$ the subgroup of $F_{4n-1}$ without the generator $y_{\gamma'}$.
  %y was once x_{4n-1}
  \item The solvability of $(R_{2n-1} (g\at{2})^x \cdot g\at{1},\gamma'|_{F_{4n-2}})$ implies
  the solvability of $(R_ng,\gamma)$.
 \end{itemize}

 We will define this map in several steps and afterwards show that for all good pairs $(q,\gamma)$ and all $g$ such that $g^\tau=q$ 
 there is some constraint $\gamma' \in \Gamma^q(\gamma)$ such that $((g\at{2})^x \cdot g\at{1},\gamma')$ is a good pair.
 
 For the first step take the branching structure $(K,Q,\pi,Q_1,\omega)$ of the Grigorchuk group as before and 
 complement the set $S$ to a transversal $S'$ of $\RNK{G}{K}$ and denote by $\textup{rep}\colon Q\to S'$ 
 the map such that $\textup{rep}(q)^\pi = q$.
 
 Denote $x_i\in X$ such that $\supp(\gamma) \subset \left<x_1,\ldots,x_{2n}\right>$ and choose some 
 other $y_1,\ldots,y_{4n} \in X \setminus \{x_1,\ldots,x_{2n}\}$. 
 Now define
 \[\Gamma_1(\gamma) = \{\gamma' \colon \left< y_1,\ldots ,y_{4n}\right> \to Q \mid \pair{\gamma'(y_{2i-1}),\gamma'(y_{2i})}\in w^{-1}(\gamma(x_i)),i=1\ldots2n\}.\] 
 Let $F_1=\left<g\right>,F_2=\left<g_1,g_2\right>$ be free groups. 
 Now define a homomorphism  
 \begin{align*}
  \Phi_{\gamma}\colon F_X*F_1 &\to (F_X*F_2)\wr C_2 \\ g&\mapsto \pair{g_1,g_2},\\ x_i\qquad &\mapsto \pair{x_i^{(1)},x_i^{(2)}}\act(\gamma(x_i)).
 \end{align*}
Take $q_1,q_2 \in Q,n\geq3\in \NN$ arbitrary and define
 \[\Gamma_2^{q_1,q_2,n}(\gamma) = \left\{\gamma'\in \Gamma_1(\gamma) \bigmid \substack{\pi\colon F_2\to Q\\
										g_1\mapsto q_1 \\
										g_2\mapsto q_2 }, (\gamma'*\pi)^2(\Phi_\gamma(R_ng))=\pair{1,1}\right\}.\] 
 Denote by $v,w=v_n,w_n$ the elements such that $\Phi_\gamma(R_ng)=\pair{v,w}\pair{g_1,g_2}$. By the following Lemma \ref{lem:commonVar} there is 
 $x_0 \in X\cup X^{-1}$ such that $v=v_1x_0v_2$ and $w=w_1x_0^{-1}w_2$. Then the homomorphism
 \[l_{x_0}\colon F_X*F_2\to F_X*F_2, x_i \mapsto \begin{cases}
						x_i &\text{ if } x_i\neq {x_0} \\
						w_2g_2w_1 &\text{ if }x_i= {x_0} 
                                             \end{cases}\]
 maps $vg_1 \mapsto v_1w_2g_2w_1v_1g_1$ and $wg_2\mapsto 1$. 
 For $\gamma'\in \Gamma_2^{q_1,q_2,n}(\gamma)$ it is $(\gamma'*\pi)(x_0)=({\gamma'*\pi})(w_2g_2w_1)$ so with $X'=X\setminus x_0$ there is no loss of information if 
 we consider $\gamma'|_{F_{X'}}$ instead of $\gamma'$.
 From section \ref{sec:normal_form} remember the normalization 
 automorphism $\Normal_{\gamma,n,x_0}:=\Normal_{v_1w_2g_2w_1v_1g_1}\colon F_{X'}*F_2 \to F_{X'}*F_2$
  and note that
 $\Normal_{\gamma,n,x_0}(l_{x_0}(v))=R_{2n-1}g_2^{{y_{\gamma'}}}g_1$ for some generator ${y_{\gamma'}}\in X$.
 This leads to the following definition.
 \[\Gamma_3^{q_1,q_2,n,x_0}(\gamma) = \left\{\gamma'|_{X'} \circ \Normal_{\gamma,n,x_0} \bigmid \gamma' \in \Gamma_2^{q_1,q_2,n} \right\}.\] 
 A solution for the the constrained equation
 \[(R_{2n-1}(g\at2)^{\textup{rep}(\gamma''({y_{\gamma''}}))}(g\at1),\gamma'')\textup{ for }\gamma'' \in \Gamma_3^{(g\at{1})^\pi,(g\at{2})^\pi,n,x_0}\]
 can be 
 extended by the map ${y_{\gamma''}}\mapsto \textup{rep}(\gamma''({y_{\gamma''}}))$ to a solution $s'$ 
 of the equation $(R_{2n-1} (g\at{2})^{{y_{\gamma'}}} g\at{1},\gamma'')$. The map $s'\circ \Normal_{\gamma,n,x_0}^{-1}$ 
 is a solution for the constrained equation $(v_1w_2g_2w_1v_1g_1,\gamma'|_{X'}:=\gamma'' \circ \Normal_{\gamma,n,x_0}^{-1})$.
 Which can be extended by the mapping $x_0\mapsto w_2(g\at{2})w_1$ to a solution $s$ of $(\Phi_\gamma(R_n g),\gamma')$.
 By definition of $\omega$ it is $t_i:=\pair{s(y_{2i-1}),s(y_{2i})}\act(\gamma(x_i)) \in G$ for all $i$. So the mapping $x_i\mapsto t_i$ is a solution for $(R_ng,\gamma)$.
 
 The map $\Gamma_3^{q_1,q_2,n,x_0}$ does not depend on the value of $n$: Assume $m<n$ then choose  
 fitting $v,w$ such that $\Phi_\gamma(R_mg)=\pair{v,w}\pair{g_1,g_2}$ then $\Phi_\gamma(R_ng)=\pair{v,w}\pair{R_{n-m}g_1,R_{n-m}'g_2}$ 
 then after applying the homomorphism $l_{x_0}$ 
 the word which needs to be normalized is $v_1w_2R_{n-m}g_2w_1v_1R_{n-m}'g_1$. The automorphisms
 \begin{align*}
 \psi_1 \colon F_X*F_2 &\to F_X*F_2, & \psi_2 \colon F_X*F_2 &\to F_X*F_2\\
 y &\mapsto y^{g_1^{-1}}, & y &\mapsto y^{g_2^{x_{11}}g_1}, & &\text{ for }y\in \Var(R_{n-m}')\\
 z &\mapsto z^{(g_2w_1v_1g_1)^{-1}}& z &\mapsto z^{g_2^{x_{11}}g_1}  & &\text{ for }z\in \Var(R_{n-m})\\
 x &\mapsto x & & & &\text{ for all other generators}
 \end{align*}
 have the property that $\Normal_{v_1w_2R_{n-m}g_2w_1v_1R_{n-m}'g_1} = \psi_2\circ\Normal_{v_1w_2g_2w_1v_1g_1}\circ \psi_1$ and 
 $\gamma' \circ \psi_i = \gamma'$. Hence $\Gamma_3^{q_1,q_2,n,x_0}$ is independent of $n$.
 
 The map $\Gamma_3^{q_1,q_2,n,x_0}$ does depend on $x_0$, therefore we take the union of all of them and define
  \[\Gamma_4^{q_1,q_2}(\gamma)\in \bigcup_{x_0\in \Var(v)\cap\Var(w)}\Gamma_3^{q_1,q_2,n,x_0}(\gamma).\]
 Note now that $q_{1},q_2\in Q$ are determined by $q\in\RNK{G'}{K'}$ in the sense that there is a map $\att{i}\colon \RNK{G'}{K'}\to Q$ such that if $g^\tau = q$ and $g_i=g\at{i}$ 
 then $q_i = q\att{i}$ (Lemma \ref{lem:atIsWellDefinedModK'}). So we can write $\Gamma_4^{q_1,q_2}$ as $\Gamma_4^q$ instead and filter those 
 constraints out which doesn't fulfill the requested properties and therefore finally define
 \[\Gamma^q(\gamma) := \left\{\gamma' \in \Gamma_4^q(\gamma) \bigmid 
  \act(\gamma')\neq 1, \gamma'({y_{\gamma'}}) \in S^\pi \right\}.\]
 \begin{pro}\label{pro:existsNextPair}
 For each good pair $(q,\gamma)$ with $q\in\RNK{G'}{K'}$ and $\gamma\in\Red_{act}$ the set $\Gamma^q(\gamma)$ 
 contains some $\gamma'$ with special generator $y_{\gamma'}$ such that for all $g$ with $g^\tau=q$ the
 pair $\left((g\at{2})^{\textup{rep}(\gamma'(y_{\gamma'})}\cdot g\at{1},\gamma'\right)$ is a good pair.
\end{pro}
\begin{proof}
 In the construction before it is clear that the sets $\Gamma_3^{q,x_0}$ are nonempty and for the finitely many $\gamma\in \Red_{act}$ it is
 easy to check whether some of the finitely many $\gamma'\in\bigcup_{x_0}\Gamma_3^{q,x_0}(\gamma)$ fulfill $\gamma'(y_\gamma) \in S^\pi$ and $\act(\gamma')\neq 1$.
 
 %As a next step we want to make sure that for all preimages $g$ of $q$ under $\tau$ there are good pairs among the
 %resulting pairs $\left((g\at{2})^{\textup{rep}(y)}\cdot g\at{1},\gamma'''\right)$.
 
  Define for $h\in G$ maps $p_h\colon G\to G$ by $g\mapsto (g\at{2})^{h}\cdot g\at{1}$ this maps are in general not homomorphisms but 
  by Lemma \ref{lem:productOfStatesIsInDerived} we see that for $g\in G'$ that $p_h(g)\in G'$ for all $h\in G$ thus there is a chance that these elements form good pairs with
  the correct choices of $\gamma' \in \Gamma^q(\gamma)$. 
 
  By Lemma \ref{lem:pIsDefinedModK'} we can define the map $\bar p_h\colon \RNK{G'}{K'} \to \RNK{G'}{(K\times K)}$
 and the natural homomorphism \[\varpi'\colon \RNK{G'}{K'} \to \RNK{\left(\RNK{G'}{K'}\right)}{\left(\RNK{K\times K}{K'}\right)} \simeq \RNK{G'}{K\times K} \]
 and now only need to show that there is a $\gamma'\in\Gamma^q(\gamma)$ such that all preimages of $\bar p_{\textup{rep}(\gamma'(y_{\gamma'})}(q)$ under $\varpi'$ 
 form good pairs with $\gamma'$. In formulas what needs to be checked is: Let $\mathcal{G}$ be the predicate of being a good pair. %TODO Find better name than \mathcal{G}
 \[\forall q\in\RNK{G'}{K'}
      \forall \gamma\in\Red_{act} 
	 \exists \gamma'\in \Gamma^q(\gamma)
	    \forall r\in \varpi'^{-1}(\bar p_{\textup{rep}(\gamma'(y_{\gamma'}))}):
	      \mathcal{G}(q,\gamma) \Rightarrow \mathcal{G}(r,\gamma').\]
 
 This last formula quantifies only over finite sets and is implemented in GAP and can be verified there by calling \lstinline{verifyPropExistsSuccessor}. 
 \end{proof}

 \begin{defi}
 For each fixed $q\in\RNK{G'}{K'}$ and $\gamma\in \Red_{act}$ such that $(q,\gamma)$ is a good pair
 fix a constraint $\gamma'\in\Gamma^q(\gamma)$ and the element $x=\textup{rep}\gamma'(y_{\gamma'})\in S$ which exist by the previous proposition.
 
 With Lemma \ref{lem:solvabilityWithReducedConstraint} we can assume that $\gamma'|_{F_{X\setminus y_{\gamma'}}}$ can be replaced by a reduced constraint $\gamma'_r$. 
 For a good pair $(g,\gamma)\in G'\times\Red_{act}$ the \emph{succesing pair} is defined as $\left((g\at{2})^xg\at{1},\gamma'_r\right)$.
 Moreover by applying this iteratively we get the \emph{succesing sequence} $(g_k,\gamma_k)$ of $(g,\gamma)$.
 \end{defi}
 
 % Keep it here for the case that an easy method to show that K' has finite c.w. comes up
 % Because then it wouldn't rely on G contracting any more
 
% \begin{cor} \label{cor:finiteCommutatorWidthKimpliesCommutatorWidth3} 
%  If $K$ has finite commutator width, this implies that the commutator width of $G$ 
%  is at most $3$.
% \end{cor}
% \begin{proof}
%  Starting with some element $g\in G'$ there is a $\gamma\in\Red_{act}$ such that
%  $(g,\gamma)$ is a good pair. (Lemma \ref{lem:existsGoodGamma}).
%  
%  By Proposition \ref{pro:existsNextPair} there exist good pairs $(g_k,\gamma_k)_{k\in\NN}$ such 
%  that $(R_3g,\gamma)$ is solvable if one (and hence all) of the constrained
%  equations $(R_{2^k+1}g_k,\gamma_k)$ is solvable.
%  
%  If $K$ is of finite commutator width then by
%  Corollary \ref{cor:finiteCommutatorWidthKimpliesBoundedConstraintedCommutators} 
%  there is an $N\in\NN$ such that all for all good pairs $(h,\gamma)$ and
%  $n\geq N$ the constrained equations $(R_ng,\gamma)$ are solvable.
%  Therefore the sequence of good pairs $(g_k,\gamma_k)$ is a sequence 
%  of solvable equations $(R_{2^k+1}g_k,\gamma_k)$.
% \end{proof}
%TODO replace y_i^\gamma_j by \gamma_j(y_i) otherwise it gets unreadable // hopefully Done everywhere. Check if its consistent!

\begin{lem} \label{lem:commonVar}
 If $\gamma$ is a constraint with nontrivial activity, and $\Phi_\gamma(R_ng)=\pair{w_1,w_2}$ then $\Var(w_1)\cap\Var(w_2)\neq \emptyset$.
\end{lem}
\begin{proof}
 Let $x$ be generator of $F_X$ with non vanishing constraint activity. 
 Then $R_n$ contains either a factor $[x,y]$ or $[y,x]$ for another generator $y$. Assume without loss of generality the first case.
 Let further be $\Phi_\gamma(x)=\pair{x_1,x_2}(1,2)$ and $\Phi_\gamma(y)=\pair{y_1,y_2}\sigma$. 
 Then $\Phi_\gamma(R_n g)$ contains a factor 
 \[ [\pair{x_1,x_2}(1,2),\pair{y_1,y_2}\sigma] = \begin{cases}
                                                   \pair{x_2^{-1}y_2^{-1}x_2y_1,x_1^{-1}y_1^{-1}x_1y_2} & \textup{ if } \sigma=\id \\
                                                   \pair{x_2^{-1}y_1^{-1}x_1y_2,x_1^{-1}y_2^{-1}x_2y_1} & \textup{ if } \sigma=(1,2) . \\
                                                 \end{cases}
\] So in both cases $y_1,y_2\in \Var(w_1)\cap \Var(w_2)$. 
\end{proof}
\begin{lem} \label{lem:productOfStatesIsInDerived} 
 Let $h\in G$ and $p_h\colon G\to G$ be the map $g\mapsto \left((g\at{2})^{h}\cdot g\at{1},\gamma'''\right)$.
 
 It holds that $p_h(G')\subset G'$ for all $h\in G$ and $p_1(K)\subset K$. 
\end{lem}
\begin{proof}
 Denote first by $p:=p_1$ then
 each element $g\in G'$ is a word in generators $w((ab)^2,(abad)^2,(bada)^2,(ad)^2)$. 
 The generators have the following form:
 \[(ab)^2=\pair{ca,ac}, (abad)^2 = \pair{1,(ab)^2}, (bada)^2=\pair{(ab)^2,1}, (ad)^2=\pair{b,b}.\]
 Therefore it is
 \begin{align*}
  p(g) &= w(ac,(ab)^2,1,b) \cdot w(ca,1,(ab)^2,b)\\ &\equiv w(ac,1,1,1) \cdot w(ca,1,1,1) \cdot w(1,1,1,b)^2 \equiv 1 \mod G'.
 \end{align*}
 For $h\in G$ it is $p_h(g)=[h,(g\at{2})^{-1}]p(g)$ and therefore $p_h(g)\in G'$ for all $g\in G'$.
 An element $g\in K$ is a word $w((ab)^2,(abad)^2,(bada)^2,)$ and therefore
 \begin{align*}
  p(g) &= w(ac,(ab)^2,1) \cdot w(ca,1,(ab)^2)\\ &\equiv w(ac,1,1) \cdot w(ca,1,1) \equiv 1 \mod K.
 \end{align*}
\end{proof}
\begin{lem} \label{lem:pIsDefinedModK'}
 The map 
 \begin{align*} 
  \bar p_h\colon \RNK{G'}{K'} &\to \RNK{G'}{K\times K}\\
  gK'&\mapsto \left(g\at{2})^h\cdot g\at{1}\right){K\times K}
 \end{align*}
is well defined.
\end{lem}
\begin{proof}
 It's easy to verify by GAP that $k\at{i} \in K\times K$ for $i=1,2$ and $k\in K'$ using Lemma \ref{lem:subgroupsOfG}.
 For check this call the function \lstinline{verifyLemmaStatesOfKPinKxK} of the attached GAP file.
 Then for $k\in K'$ it is 
 \[p_h(gk) = ((gk)\at{2})^h\cdot (gk)\at{1} = (g\at{2})^h\cdot (k\at{2})^h\cdot g\at{1}\cdot k\at{1}\in (g\at{2})^h\cdot g\at{1}K\times K.\]
\end{proof}
\begin{lem} \label{lem:atIsWellDefinedModK'}
 The maps $\at{i}\colon G\to G, g\mapsto g\at{i}$ induce well defined maps $\att{i}\colon \RNK{G}{K'}\to\RNK{G}{K}$
\end{lem}
\begin{proof}
 As before it is $k'\at{i} \in K\times K < K$.
%  
%  Note that $\at{i}|_{G'}$ is a group homomorphism then consider the following diagram:
%  
%  \hfill
%  \begin{xy}
%   \xymatrix{
%       1 \ar[r]& \RNK{K}{K'} \ar[r] \ar[d]_\varphi   & \RNK{G'}{K'}   \ar[r] & \RNK{G'}{K} \ar[r]\ar[d]_=& 1  \\
%       1 \ar[r]& \RNK{K}{K\times K} \ar[r]   & \RNK{G'}{K\times K}   \ar[r] & \RNK{G'}{K} \ar[r] &1  
%   }
% \end{xy}\hfill\ 
% 
% Where the map $\varphi$ exists because the group $\RNK{K}{K\times K}$ has order $4$ and hence is abelian. So there need to be a homomorphism 
% $\RNK{G'}{K'}\to \RNK{G'}{K\times K}$ which makes all cells commute.
\end{proof}
\subsection{Product of 3 commutators}
We will prove that every element $g\in G'$ is a product of three commutators by proving that all
sequences $(g_k,\gamma_k)$ as defined after Proposition \ref{pro:existsNextPair} are finite.
For this purpose remember the map $p_x\colon g \mapsto (g\at{2})^x g\at{1}$ from the proof of Proposition \ref{pro:existsNextPair}.
We will show that for each $g\in G'$ the sequence of sets 
\[\textup{Suc}_1^g=\{g\},\ \textup{Suc}_n^g = \{p_x(h) \mid h\in \textup{Suc}_{n-1}^g, x\in S\} \]
stagnates in a finite set. 

In \cite{Bartholdi:Growth} there is a choice of weights on generators which result in a length on $G$ with good properties.
\begin{lem}[\cite{Bartholdi:Growth}] \label{lem:laurentsweights}
 Let $\eta\approx 0.811$ be the real root of $x^3+x^2+x-2$ and set the weights 
 \begin{align*}
  \omega(a) &= 1-\eta^3 & \omega(c)&=1-\eta^2 \\ \omega(b)&= \eta^3 & \omega(d)&=1-\eta
 \end{align*}
 then 
 \begin{align*}
  \eta(\omega(b)+\omega(a)) &= \omega(c)+\omega(a) \\
  \eta(\omega(c)+\omega(a)) &= \omega(d)+\omega(a) \\
  \eta(\omega(d)+\omega(a)) &= \omega(b).
 \end{align*}
\end{lem}
The next lemma is a small variation of a lemma in \cite{Bartholdi:Growth}.
\begin{lem}
 Denote by $\partial_\omega$ the length on $G$ induced by the weight $\omega$. Then
 $\partial_\omega(p_x(g)) \leq \delta \partial_\omega(g)$ for all $x\in S, g\in G$ with $\partial_\omega(g)>C$ some constant $C\in\NN,\delta<1$.
\end{lem}
\begin{cor}
The sequences of sets
 \[\textup{Suc}_1^g=\{g\},\ \textup{Suc}_n^g = \{p_x(h) \mid h\in \textup{Suc}_{n-1}^g, x\in S\} \]
 stagnates at a finite step for all $g\in G$.
\end{cor}
\begin{proof}[Proof of Lemma.(\cite{Bartholdi:Growth})] 
 Each element $g\in G$ can be written in a word of minimal length of the form $g=a^\varepsilon x_1 a x_2 a\ldots x_n a^\delta$ where
 $x_i\in \{b,c,d\}$ and $\varepsilon,\delta\in \{0,1\}$. Denote by $n_b,n_c,n_d$ the number of occurrences of $b,c,d$ accordingly. 
 Then
 \begin{align*}
  \partial_\omega(g) &= (n-1+\varepsilon+\delta)\omega(a)+n_b\omega(b)+n_c\omega(c)+n_d\omega(d)\\
  \partial_\omega(p_x(g)) &\leq (n_b+n_c)\omega(a)+n_b\omega(c)+n_c\omega(d)+n_d\omega(b) + 2\partial_\omega(x)\\
  &= \eta\left( (n_b+n_c+n_d)\omega(a)+n_b\omega(b)+n_c\omega(c)+n_d\omega(d) \right) + 2\partial_\omega(x)\\
  &= \eta(\partial_\omega(g) +(1-\varepsilon-\delta)\omega(a)) + 2\partial_\omega(x) \\
  &\leq \eta(\partial_\omega(g)+\omega(a)) + 2(\omega(a)+\omega(b))\\
  &= \eta(\partial_\omega(g)+\omega(a)) + 2.
 \end{align*}
 Thus the length of $p_x(g)$ growths with a linear factor smaller $1$ in terms of the length of $g$. Therefore the claim holds.
 For instance one could take $\delta =0.86$ and $C=50$ or $\delta=0.96$ and $C=16$.
\end{proof}
This completes the proof of the following proposition
\begin{pro}\label{pro:solvableConstraintedEquations}
 If $n\geq3$ and $(g,\gamma)$ is a good pair with active constraint $\gamma$ with $\supp(\gamma)\subset\{x_1,\ldots,x_{2n}\}$
 then the constrained equation $(R_n(x_1,\ldots,x_{2n})g,\gamma)$ is solvable. 
\end{pro}
\begin{cor}
 The Grigorchuk group $G$ has commutator width at most $3$.
\end{cor}
\begin{proof}
 This is a direct consequence of the proposition and Lemma \ref{lem:existsGoodGamma}.
\end{proof}

\subsection{Product of 2 commutators}
The case of products of two commutators can be reduced to the case of three commutators by using the same method as before.

We can compute the orbits of $\RNK{\Aut(F_4)}{\Stab(R_2)}$ and take a representative system denoted by $\Red^4$.
It turns out that there are $86$ orbits and we can check that there are again enough active constraints:
\begin{lem} \label{lem:existsGoodGammaForRed4}
 For each $q\in\RNK{G'}{K'}$ there is $\gamma\in \Red^4_{act}$ such that $(q,\gamma)$ is a 
 good pair.
\end{lem}
\begin{proof}
 This can be checked in GAP with the function \lstinline{verifyLemmaExistGoodGammasForRed4}.
\end{proof}

Analog to Proposition \ref{pro:existsNextPair} we can formulate the following proposition.
\begin{pro}\label{pro:existsNextPair4}
 For each good pair $(q,\gamma)$ with $q\in\RNK{G'}{K'}$ and $\gamma\in\Red^4_{act}$ the set $\Gamma^q(\gamma)$ 
 contains some $\gamma'$ with special generator $y_{\gamma'}$ such that for all $g$ with $g^\tau=q$ the
 pair $\left((g\at{2})^{\textup{rep}(\gamma'(y_{\gamma'})}\cdot g\at{1},\gamma'\right)$ is a good pair.
\end{pro}
\begin{proof}
The proof is the same as for Proposition \ref{pro:existsNextPair}. The corresponding formula which needs to be checked is 
\[\forall q\in\RNK{G'}{K'}
      \forall \gamma\in\Red^4_{act} 
	 \exists \gamma'\in \Gamma^q(\gamma)
	    \forall r\in \varpi'^{-1}(\bar p_{\textup{rep}(\gamma'(y_{\gamma'}))}):
	      \mathcal{G}(q,\gamma) \Rightarrow \mathcal{G}(r,\gamma').\]
 With GAP and the function \lstinline{verifyPropExistsSuccessor} this can be verified. 
\end{proof}
The resulting succesing pairs are now equations of genus $3$ with an active constraint. Those are already shown to be solvable 
by Proposition \ref{pro:solvableConstraintedEquations}. So together with Lemma \ref{lem:existsGoodGammaForRed4} this proves the following corollary.
\begin{cor}
 The Grigorchuk group $G$ has commutator width at most $2$.
\end{cor}
\begin{cor}
 $K$ has commutator width at most $2$. 
\end{cor}
\begin{proof}
 To show that $K$ has commutator width $2$ it is sufficient, to show that the constrained equations $(R_2g,\id)$ have solutions for all $g\in K'$. 
 Since $\id$ has trivial activity one cannot directly apply Proposition \ref{pro:solvableConstraintedEquations}.
 But one can check that all pairs $(h,\gamma_1),(f,\gamma_2)$
 such that $g=\pair{h,f}$ and $\gamma_1=(1,1,(bad)^\pi,1)$, $\gamma_2=(1,1,1,(ca)^\pi)$
 are good pairs with active constraints and hence admit solutions $s_1,s_2\colon F_4\to G$.
 
 Then we can define the map $s\colon F_4\to G, x_i\mapsto \pair{s_1(x_i),s_2(x_i)}$ this is a solution
 for $R_2g$ and $s(x_i)\in K$ for all $i=1,\ldots,4$. Therefore the commutator width of $K$ is at most $2$.
 
 The previous check is implemented in the GAP function \lstinline{verifyCorollaryFiniteCWK}. 
\end{proof}
\subsection{Not every element is a commutator}
The previous procedure can not be used to prove that each element is a commutator since for equations of genus $1$ the 
genus does not increase by passing to a succesing pair. 

In fact not every element $g\in G'$ is a commutator. This can be seen by passing to finite quotients. If every element would be a commutator then
it would be a commutator in the quotient group. 
%For example the element $d(ac)^2ada$ is not a commutator. TODO this is not an example. Find one
This element acts on the %nth
level on the tree like the permutation %()
But is not a commutator in $\RNK{G}{\Stab(n)}$.

%TODO find out the permutations.


\begin{cor}
 All elements $g\in G'$ are products of two commutators.
\end{cor}

\section{Implementation in GAP}
\subsection{Usage of the attached files}
Together with this document there come some files which contain the algorithms used for some proofs.
The file \emph{verify.g} is meant to be the starting point. The file contains the main methods to explore the results.
After reading this file the following functions can be used:
\begin{itemize}
 \item \lstinline{ReducedConstraint}: Given a group homomorphism $\varphi\colon F_{2n} \to Q$ this function returns a reduced constraint.
 Example: 
 \begin{lstlisting}
    f1 := Q.1; f2 := Q.2;
    gamma:= [f1,f1,f1,f1,f1,f1];
    constr := ReducedConstraint(gamma);;
    Print(constr.constraint);
    gamma := GroupHomomorphismByImages(FreeGroup(6),Q,[f2,f2,f2,f2,f2,f2]);
    constr := ReducedConstraint(gamma);;
    Print(constr.constraint);
\end{lstlisting} 
\end{itemize}

\subsection{Implementation details}
\paragraph{Lemma \ref{lem:atIsWellDefinedModK'}}
\paragraph{Lemma \ref{lem:existsGoodGamma'}}

\phantomsection
\addcontentsline{toc}{section}{References}
\bibliography{latex/bio}
\appendix
% \include{Abschnitte/Anhang}
%\include{Abschnitte/Erklaerung}


\end{document}
