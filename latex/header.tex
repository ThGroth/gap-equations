\usepackage[utf8]{inputenc}
\usepackage[T1]{fontenc}
\usepackage{setspace}
%\usepackage[ngerman]{babel}
\usepackage{lmodern}


\usepackage{amsmath}
\usepackage{amsthm}
\usepackage{amssymb}
% Boldface Zahlen
\usepackage{bbm}
\usepackage{enumerate}
\usepackage{mathtools}

%For include without pagebreak
\usepackage{newclude}

% For relative Textsizes
\usepackage{relsize}

%For units
\usepackage[binary-units=true]{siunitx}

\usepackage[disable]{todonotes}

\usepackage{graphicx}
% Bibliographi
%\usepackage{natbib}

%Für Quellcode
\usepackage{listings}

%Für Verlinkungen im Dokument
\usepackage{xcolor}
\definecolor{white}{rgb}{1,1,1}
\usepackage[linkbordercolor=white,urlbordercolor=white ,citebordercolor=white, plainpages=false]{hyperref}

%Zum Erstellen von Graphen für Automaten
\usepackage{tikz}
\usetikzlibrary{arrows,shapes.geometric,automata,positioning}
\tikzset{elliptic state/.style={draw,ellipse}}

%Diagrams
\usepackage[arrow, matrix, curve]{xy}


%For extra small fraction \sfrac 
\usepackage{xfrac}

%Reset the equation counter after each subsection
\usepackage{chngcntr}
\counterwithin*{equation}{section}
\counterwithin*{equation}{subsection}
%Für Zeilennummern
%\usepackage[modulo]{lineno}
%\linenumbers

%Counter für Automaten
\newcounter{automatonnumber}[section]
\renewcommand{\theautomatonnumber}{\mathcal{A}_{\thesection,\arabic{automatonnumber}}}
\newcommand{\autno}[1]{
  \refstepcounter{automatonnumber}
  \label{#1}
  (\theautomatonnumber)
}

%Quellcode Aussehen:
\lstset {
basicstyle=\footnotesize,
breaklines=true,
showstringspaces=false,
tabsize=2
}

\newtheorem{pro}{Proposition}[section]
\newtheorem{con}[pro]{Conjecture}
\newtheorem{thm}[pro]{Theorem}
\newtheorem{cor}[pro]{Corollary}
\newtheorem{lem}[pro]{Lemma}

\newtheorem{task}{Task}

\theoremstyle{definition}
\newtheorem*{ex}{Example}
\newtheorem*{re}{Remark}
\newtheorem{defi}[pro]{Definition}
\newtheorem{cordef}[pro]{Corolary/Definition}


%Für Quotienten
\newcommand{\RRNK}[2]{
  \raisebox{1ex}{\ensuremath{#1}}
  \ensuremath{\mkern-3mu} \bigg/  \ensuremath{\mkern-3mu}
  \raisebox{-1ex}{\ensuremath{#2}}
}
\newcommand{\RNK}[2]{{#1/#2}}
% \newcommand{\RNK}[2]{
%   \mathchoice{ \raisebox{1ex}{\ensuremath{#1}}%
% 	       \ensuremath{\mkern-4mu} \big/  \ensuremath{\mkern-4mu}%
% 	       \raisebox{-1ex}{\ensuremath{#2}}}%
% 	     { \raisebox{0.2ex}{\ensuremath{#1}}%
% 	       \ensuremath{\mkern-4mu} /  \ensuremath{\mkern-4mu}%
% 	       \raisebox{-0.2ex}{\ensuremath{#2}}}%
% 	     { \raisebox{0.1ex}{\ensuremath{\scriptstyle#1}}%
% 	       \ensuremath{\mkern-4mu} /  \ensuremath{\mkern-4mu}%
% 	       \raisebox{-0.1ex}{\ensuremath{\scriptstyle#2}}}%
% 	     { {\ensuremath{#1}}%
% 	        /%
% 	       {\ensuremath{#2}}}%
%   \ 
% }

%uppergausian bracket
\DeclarePairedDelimiter{\ceil}{\lceil}{\rceil}

%Pfeile für exakte Sequenzen
\newcommand{\exar}[1][ ]{\overset{#1}{\longrightarrow}}
%Normalteiler/Ideal 
\newcommand{\normal}{{\mathrel\trianglelefteq}}

%großer Strich
\newcommand{\bigmid}{\ \middle \vert }

%Besondere Buchstaben
\newcommand{\NN}{\mathbb{N}}
\newcommand{\ZZ}{\mathbb{Z}}
\newcommand{\QQ}{\mathbb{Q}}
\newcommand{\RR}{\mathbb{R}}
\newcommand{\CC}{\mathbb{C}}
\newcommand{\HH}{\mathbb{H}}
\newcommand{\PP}{\mathbb{P}}
\newcommand{\one}{\mathbbm{1}}

%das d hinter dem Integral
\newcommand{\intd}{d}

%Sprachen
\newcommand{\La}{\mathcal{L}}
\newcommand{\emptyword}{\epsilon}
%Zustandmenge
\newcommand{\States}{\zeta}
%Alphabet
\newcommand{\Al}{\mathcal{A}}
%Zentrum
\newcommand{\Ce}{\mathfrak{C}}
%Free Variables in a groupword
\newcommand{\Var}{\textup{Var}}
%Commutator width
\newcommand{\cw}{\textup{width}}
%Stabilizer
\newcommand{\Sta}{\textup{St}}
%Projektion auf Alphabet
\newcommand{\alphab}{\mathfrak{wr}}
\newcommand{\state}{\mathfrak{state}} 
%zugehörige Mealy
\newcommand{\mealy}[1]{\mathfrak{M}_{#1}}
%Aktivität eines Automorphismus
%\newcommand{\act}{\mathcal{A}\mathcal{CT}}
%\newcommand{\act}{{\scalebox{0.9}{$\mathcal{A}$}\mspace{-2mu}\raisebox{-0.3ex}{\tiny$\mathbf{ct}$}}}
\newcommand{\act}{{\textup{\smaller$\mathcal{A}\mathbf{ct}$}}}
%PairKlammern
\DeclarePairedDelimiter{\pair}{\langle\negthickspace\langle}{\rangle\negthickspace\rangle}
%\newcommand{\pair}[1]{{\langle\negthickspace\langle #1 \rangle\negthickspace\rangle}}
%Nucleus
\newcommand{\Nuc}{\mathcal{N}}
%Commutatror Width
\newcommand{\mcount}{{\#_M}}
%Normal normal_form
\newcommand{\Normal}{\mathfrak{nf}}
%Reduced constraints
\newcommand{\Red}{\mathfrak{R}}
%Ideale
\newcommand{\ai}{\mathfrak{a}}
\newcommand{\bi}{\mathfrak{b}}
\newcommand{\Ji}{\mathcal{Ji}}

%Zustände von Automaten

%\newcommand{\at}[1]{  {\raisebox{-0.3ex}[1ex]{$\mathbin{@}$}#1} }
\newcommand{\at}[1]{  {{\textup{\smaller @}}#1} }
\newcommand{\att}[1]{  {{\bar{\textup{\smaller @}}}#1} }
%\newcommand{\at}[1]{  |_{#1} }

%Centralizer:
\newcommand{\centra}{\mathcal{C}}
%Stabilizer
\newcommand{\Stab}{\textup{Stab}}
%Undefined elememts
\newcommand{\undef}{\varnothing}

%Funktionen aufrecht
\newcommand{\Aut}{\textup{Aut}}
\newcommand{\RAut}{\textup{RAut}}
\newcommand{\FAut}{\textup{FAut}}
\newcommand{\Poly}{\textup{Poly}}
\newcommand{\EPoly}{\mbox{\smaller$\frac{1}{2}$}\Poly}
%\newcommand{\EPoly}{\textup{EPoly}}
\newcommand{\Orb}{\textup{Orb}}
\newcommand{\OS}{\textup{OS}}
\newcommand{\OSA}{\textup{OSA}}
\newcommand{\supp}{\textup{supp}}
\newcommand{\End}{\textup{End}}
\newcommand{\rad}{\textup{rad}}
\newcommand{\ad}{\textup{ad}}
\newcommand{\id}{\mathbbm{1}}
\newcommand{\im}{\textup{i}}
\newcommand{\Min}{\textup{Min}}
\newcommand{\Imm}{\textup{Im}}
\newcommand{\Span}{\textup{span}}
\newcommand{\Spec}{\textup{Spec}}
\newcommand{\Spur}{\textup{tr}}
\newcommand{\rk}{\textup{rk}}
\newcommand{\sign}{\textup{sign}}
\newcommand{\diag}{\textup{diag}}
\newcommand{\Sym}{\textup{Sym}}
\newcommand{\ord}{\textup{ord}}
\newcommand{\suc}{\textup{suc}}
\DeclareMathOperator{\lcm}{lcm}
\DeclareMathOperator{\operp}{{ \bigcirc\!\!\!\!\!\!\!\perp}}

\newcommand{\BIGOP}[1]{\mathop{\mathchoice%
{\raise-0.22em\hbox{\huge $#1$}}%
{\raise-0.05em\hbox{\Large $#1$}}{\hbox{\large $#1$}}{#1}}}
\newcommand{\bigtimes}{\BIGOP{\times}}

\newcommand{\bigcomp}{\BIGOP{\circ}}

% nur fuer Bigboxplus andere Korrekturen
\newcommand{\BIGboxplus}{\mathop{\mathchoice%
{\raise-0.35em\hbox{\huge $\boxplus$}}%
{\raise-0.15em\hbox{\Large $\boxplus$}}{\hbox{\large $\boxplus$}}{\boxplus}}}
\parindent=0em
