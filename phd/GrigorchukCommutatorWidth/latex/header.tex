\usepackage[utf8]{inputenc}
\usepackage[T1]{fontenc}
\usepackage{setspace}
\usepackage{lmodern}


\usepackage{amsmath}
\usepackage{amsthm}
\usepackage{amssymb}
% boldface numbers
\usepackage{bbm}
\usepackage{enumerate}
\usepackage{mathtools}

%colors with names
\usepackage[x11names]{xcolor}

%For include without pagebreak
\usepackage{newclude}

% For relative textsizes
\usepackage{relsize}

\usepackage{graphicx}

%For multiline captions
\usepackage[skip=2pt]{caption}

%for sources
\usepackage{listings}

\usepackage[linkbordercolor=white,urlbordercolor=white ,citebordercolor=white, plainpages=false]{hyperref}

\usepackage{tikz}
\usetikzlibrary{arrows,shapes.geometric,automata,positioning}
\tikzset{elliptic state/.style={draw,ellipse}}

%Strikeout in math mode
\usepackage{cancel}

%For extra small fraction \sfrac 
\usepackage{xfrac}

%Reset the equation counter after each subsection
\usepackage{chngcntr}
\counterwithin*{equation}{section}
\counterwithin*{equation}{subsection}

\usepackage{textcomp}
\usepackage[space=true]{accsupp}
% requires the latest version of package accsupp
\newcommand{\copyablespace}{
    \BeginAccSupp{method=hex,unicode,ActualText=00A0}
\ %
    \EndAccSupp{}
}
%Appearance of gap sources:
\lstset {
  upquote=true,
  columns=fullflexible,
  literate={*}{{\char42}}1
         {-}{{\char45}}1,
  basicstyle=\normalsize\ttfamily,
  language=GAP,
 % showstringspaces=false,
 % tabsize=2,
}
%Monospace in inline code
%\lstMakeShortInline[columns=fixed]|

\newtheorem{pro}{Proposition}[section]
\newtheorem{con}[pro]{Conjecture}
\newtheorem{thm}[pro]{Theorem}
\newtheorem{thma}{Theorem}
\renewcommand{\thethma}{\Alph{thma}}
\newtheorem{cor}[pro]{Corollary}
\newtheorem{cora}[thma]{Corollary}
\renewcommand{\thecora}{\Alph{cora}}
\newtheorem{lem}[pro]{Lemma}

\newtheorem{task}{Task}

\theoremstyle{definition}
\newtheorem{ex}{Example}
\newtheorem*{re}{Remark}
\newtheorem{defi}[pro]{Definition}
\newtheorem{nota}[pro]{Notation}
\newtheorem{cordef}[pro]{Corollary/Definition}


%for quotients
\newcommand{\RRNK}[2]{
  \raisebox{1ex}{\ensuremath{#1}}
  \ensuremath{\mkern-3mu} \bigg/  \ensuremath{\mkern-3mu}
  \raisebox{-1ex}{\ensuremath{#2}}
}
\newcommand{\RNK}[2]{{#1/#2}}

%uppergausian bracket
\DeclarePairedDelimiter{\ceil}{\lceil}{\rceil}

%normal divisor, ideal
\newcommand{\normal}{{\mathrel\trianglelefteq}}

%großer Strich
\newcommand{\bigmid}{\ \middle \vert }

%bb symbols
\newcommand{\NN}{\mathbb{N}}
\newcommand{\ZZ}{\mathbb{Z}}
\newcommand{\QQ}{\mathbb{Q}}
\newcommand{\RR}{\mathbb{R}}
\newcommand{\CC}{\mathbb{C}}
\newcommand{\HH}{\mathbb{H}}
\newcommand{\PP}{\mathbb{P}}
\newcommand{\one}{\mathbbm{1}}

%Integrate d 
\newcommand{\intd}{d}

\newcommand{\emptyword}{\epsilon}

%Alphabet
\newcommand{\Al}{\mathcal{A}}

%Free Variables in a groupword
\newcommand{\Var}{\textup{Var}}
%Commutator width
\DeclareMathOperator{\cw}{width}
%Stabilizer
\DeclareMathOperator{\Stab}{Stab}
%Activity
\DeclareMathOperator{\act}{act}
%Rep
\DeclareMathOperator{\rep}{rep}
%Germgroup
\DeclareMathOperator{\germ}{germ}
%Pair
\DeclarePairedDelimiter{\pair}{\langle\negthickspace\langle}{\rangle\negthickspace\rangle}
%Nucleus
\newcommand{\Nuc}{\mathcal{N}}
%Normal_form
\newcommand{\nf}{\mathfrak{nf}}
%Reduced constraints
\newcommand{\Red}{\mathfrak{R}}

%States:
%\newcommand{\at}[1]{  {\raisebox{-0.3ex}[1ex]{$\mathbin{@}$}#1} }
\newcommand{\at}[1]{  {{\textup{\smaller @}}#1} }
\newcommand{\att}[1]{  {{\bar{\textup{\smaller @}}}#1} }
%\newcommand{\at}[1]{  |_{#1} }

%Centralizer:
\newcommand{\centra}{\mathcal{C}}
%Undefined elememts
\newcommand{\undef}{\varnothing}

%Variable names for equations
\newcommand{\eq}[1]{\mathcal{#1}}
\newcommand{\im}{\textup{i}}

%Identity
\newcommand{\id}{\mathbbm{1}}
%

\DeclareMathOperator{\Aut}{{Aut}}
\DeclareMathOperator{\RAut}{{RAut}}
\DeclareMathOperator{\FAut}{{FAut}}
\DeclareMathOperator{\Poly}{{Poly}}
\DeclareMathOperator{\EPoly}{\mbox{\smaller$\frac{1}{2}$}\Poly}
\DeclareMathOperator{\Orb}{{Orb}}
\DeclareMathOperator{\OS}{{OS}}
\DeclareMathOperator{\OSA}{{OSA}}
\DeclareMathOperator{\supp}{{supp}}
\DeclareMathOperator{\End}{{End}}
\DeclareMathOperator{\rad}{{rad}}
\DeclareMathOperator{\ad}{{ad}}

\DeclareMathOperator*{\Min}{{Min}}
\DeclareMathOperator{\Imm}{{Im}}
\DeclareMathOperator{\Span}{{span}}
\DeclareMathOperator{\Spec}{{Spec}}
\DeclareMathOperator{\Spur}{{tr}}
\DeclareMathOperator{\rk}{{rk}}
\DeclareMathOperator{\sign}{{sign}}
\DeclareMathOperator{\diag}{{diag}}
\DeclareMathOperator{\Sym}{{Sym}}
\DeclareMathOperator{\ord}{{ord}}
\DeclareMathOperator{\suc}{{suc}}
\DeclareMathOperator{\lcm}{lcm}
\DeclareMathOperator{\operp}{{ \bigcirc\!\!\!\!\!\!\!\perp}}

\newcommand{\BIGOP}[1]{\mathop{\mathchoice%
{\raise-0.22em\hbox{\huge $#1$}}%
{\raise-0.05em\hbox{\Large $#1$}}{\hbox{\large $#1$}}{#1}}}
\newcommand{\bigtimes}{\BIGOP{\times}}
\newcommand{\bigcomp}{\BIGOP{\circ}}


% For filenames
\newcommand{\filename}[1]{{\texttt{#1}}}
%\DeclareTextFontCommand{\gapinline}{\smaller\myfont}
\newcommand{\gapinline}[1]{{\lstinline{#1}}}
