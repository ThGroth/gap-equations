% generated by GAPDoc2LaTeX from XML source (Frank Luebeck)
\documentclass[a4paper,11pt]{report}

\usepackage{a4wide}
\sloppy
\pagestyle{myheadings}
\usepackage{amssymb}
\usepackage[latin1]{inputenc}
\usepackage{makeidx}
\makeindex
\usepackage{color}
\definecolor{FireBrick}{rgb}{0.5812,0.0074,0.0083}
\definecolor{RoyalBlue}{rgb}{0.0236,0.0894,0.6179}
\definecolor{RoyalGreen}{rgb}{0.0236,0.6179,0.0894}
\definecolor{RoyalRed}{rgb}{0.6179,0.0236,0.0894}
\definecolor{LightBlue}{rgb}{0.8544,0.9511,1.0000}
\definecolor{Black}{rgb}{0.0,0.0,0.0}

\definecolor{linkColor}{rgb}{0.0,0.0,0.554}
\definecolor{citeColor}{rgb}{0.0,0.0,0.554}
\definecolor{fileColor}{rgb}{0.0,0.0,0.554}
\definecolor{urlColor}{rgb}{0.0,0.0,0.554}
\definecolor{promptColor}{rgb}{0.0,0.0,0.589}
\definecolor{brkpromptColor}{rgb}{0.589,0.0,0.0}
\definecolor{gapinputColor}{rgb}{0.589,0.0,0.0}
\definecolor{gapoutputColor}{rgb}{0.0,0.0,0.0}

%%  for a long time these were red and blue by default,
%%  now black, but keep variables to overwrite
\definecolor{FuncColor}{rgb}{0.0,0.0,0.0}
%% strange name because of pdflatex bug:
\definecolor{Chapter }{rgb}{0.0,0.0,0.0}
\definecolor{DarkOlive}{rgb}{0.1047,0.2412,0.0064}


\usepackage{fancyvrb}

\usepackage{mathptmx,helvet}
\usepackage[T1]{fontenc}
\usepackage{textcomp}


\usepackage[
            pdftex=true,
            bookmarks=true,        
            a4paper=true,
            pdftitle={Written with GAPDoc},
            pdfcreator={LaTeX with hyperref package / GAPDoc},
            colorlinks=true,
            backref=page,
            breaklinks=true,
            linkcolor=linkColor,
            citecolor=citeColor,
            filecolor=fileColor,
            urlcolor=urlColor,
            pdfpagemode={UseNone}, 
           ]{hyperref}

\newcommand{\maintitlesize}{\fontsize{50}{55}\selectfont}

% write page numbers to a .pnr log file for online help
\newwrite\pagenrlog
\immediate\openout\pagenrlog =\jobname.pnr
\immediate\write\pagenrlog{PAGENRS := [}
\newcommand{\logpage}[1]{\protect\write\pagenrlog{#1, \thepage,}}
%% were never documented, give conflicts with some additional packages

\newcommand{\GAP}{\textsf{GAP}}

%% nicer description environments, allows long labels
\usepackage{enumitem}
\setdescription{style=nextline}

%% depth of toc
\setcounter{tocdepth}{1}





%% command for ColorPrompt style examples
\newcommand{\gapprompt}[1]{\color{promptColor}{\bfseries #1}}
\newcommand{\gapbrkprompt}[1]{\color{brkpromptColor}{\bfseries #1}}
\newcommand{\gapinput}[1]{\color{gapinputColor}{#1}}


\begin{document}

\logpage{[ 0, 0, 0 ]}
\begin{titlepage}
\mbox{}\vfill

\begin{center}{\maintitlesize \textbf{\textsf{Equations}\mbox{}}}\\
\vfill

\hypersetup{pdftitle=\textsf{Equations}}
\markright{\scriptsize \mbox{}\hfill \textsf{Equations} \hfill\mbox{}}
{\Huge Version 0.1.2\mbox{}}\\[1cm]
{15 March 2017\mbox{}}\\[1cm]
\mbox{}\\[2cm]
{\Large \textbf{ Thorsten Groth  \mbox{}}}\\
\hypersetup{pdfauthor= Thorsten Groth  }
\end{center}\vfill

\mbox{}\\
{\mbox{}\\
\small \noindent \textbf{ Thorsten Groth  }  Email: \href{mailto://thorsten.groth@mathematik.uni-goettingen.de} {\texttt{thorsten.groth@mathematik.uni-goettingen.de}}}\\
\end{titlepage}

\newpage\setcounter{page}{2}
{\small 
\section*{Copyright}
\logpage{[ 0, 0, 1 ]}
 \index{License} {\copyright} 2016 by Thorsten Groth

 \textsf{Equations} package is free software; you can redistribute it and/or modify it under the
terms of the \href{http://www.fsf.org/licenses/gpl.html} {GNU General Public License} as published by the Free Software Foundation; either version 2 of the License,
or (at your option) any later version. 

 This program is distributed WITHOUT ANY WARRANTY; without even the implied
warranty of MERCHANTABILITY or FITNESS FOR A PARTICULAR PURPOSE. See the GNU
General Public License for more details. \mbox{}}\\[1cm]
\newpage

\def\contentsname{Contents\logpage{[ 0, 0, 2 ]}}

\tableofcontents
\newpage

  
\chapter{\textcolor{Chapter }{Installation}}\logpage{[ 1, 0, 0 ]}
\hyperdef{L}{X8360C04082558A12}{}
{
 

 The package is installed by unpacking the archive in the \texttt{pkg/} directory of your \textsf{GAP} installation. 
\begin{Verbatim}[commandchars=!@|,fontsize=\small,frame=single,label=Example]
  !gapprompt@gap>| !gapinput@LoadPackage("equations");|
  true
\end{Verbatim}
 }

 
\chapter{\textcolor{Chapter }{Example Session}}\logpage{[ 2, 0, 0 ]}
\hyperdef{L}{X86BBB40E7FB1245E}{}
{
 We show some examples for using this package. The used methods are described
in the latter chapter. 
\section{\textcolor{Chapter }{Normal form of equations}}\logpage{[ 2, 1, 0 ]}
\hyperdef{L}{X7CF7C60A85F02022}{}
{
 Let us consider some equations over the alternating group $\textup{A}_5$. We start with defining the group in which our equations live in: 
\begin{Verbatim}[commandchars=!@|,fontsize=\small,frame=single,label=Example]
  !gapprompt@gap>| !gapinput@LoadPackage("equations");|
  true
  !gapprompt@gap>| !gapinput@A5 := AlternatingGroup(5);;SetName(A5,"A5"); |
  !gapprompt@gap>| !gapinput@F := FreeGroup(3,"X");;SetName(F,"F");|
  !gapprompt@gap>| !gapinput@EqG := EquationGroup(A5,F);|
  A5*F
   
\end{Verbatim}
 Now we enter the equation $E=X_2(1,2,3)X_1^{-1}X_2^{-1}(1,3)(4,5)X_3X_1X_3^{-1}$. 
\begin{Verbatim}[commandchars=!@|,fontsize=\small,frame=single,label=Example]
  !gapprompt@gap>| !gapinput@g := (1,2,3);;h := (1,3)(4,5);;|
  !gapprompt@gap>| !gapinput@eq := Equation(EqG,[F.2,g,F.1^-1*F.2^-1,h,F.3*F.1*F.3^-1]);|
  Equation in [ X1, X2, X3 ]
  !gapprompt@gap>| !gapinput@Print(eq);|
  FreeProductElm([ X2, (1,2,3), X1^-1*X2^-1, (1,3)(4,5), X3*X1*X3^-1 ])
   
\end{Verbatim}
 Let us see what the normal form of this equation is: 
\begin{Verbatim}[commandchars=!@|,fontsize=\small,frame=single,label=Example]
  !gapprompt@gap>| !gapinput@Genus(eq);|
  1
  !gapprompt@gap>| !gapinput@Nf := EquationNormalForm(eq);;|
  !gapprompt@gap>| !gapinput@Print(Nf.nf);|
  FreeProductElm([ X1^-1*X2^-1*X1*X2*X3^-1, (1,2,3), X3, (1,3)(4,5) ])
   
\end{Verbatim}
 We know a solution for this normal form: $s\colon X_1 \mapsto (1,2,4),\ X_2 \mapsto (1,2,5),\ X_3\mapsto ()$. 
\begin{Verbatim}[commandchars=!@|,fontsize=\small,frame=single,label=Example]
  !gapprompt@gap>| !gapinput@s:=EquationEvaluation(EqG,[F.1,F.2,F.3],[(1,2,4),(1,2,5),()]);|
  [ X1, X2, X3 ]"->"[ (1,2,4), (1,2,5), () ]
  !gapprompt@gap>| !gapinput@IsSolution(s,Nf.nf);|
  true
  !gapprompt@gap>| !gapinput@Nf.nf^s;|
  ()
  !gapprompt@gap>| !gapinput@IsSolution(s,eq);|
  false
  !gapprompt@gap>| !gapinput@eq^s;|
  (1,2,4,3,5)
   
\end{Verbatim}
 Let us compute the solution for $E$. 
\begin{Verbatim}[commandchars=!@|,fontsize=\small,frame=single,label=Example]
  !gapprompt@gap>| !gapinput@sE:= Nf.hom*s;;|
  !gapprompt@gap>| !gapinput@IsSolution(sE,eq);|
  true;
  List([1,2,3],i->ImageElm(sE,F.(i)));
  [ (2,3,4), (), (1,2,5,4,3) ]
   
\end{Verbatim}
 Thus $s_E\colon X_1 \mapsto (2,3,4),\ X_2 \mapsto (),\ X_3\mapsto (1,2,5,4,3)$ is a solution for the equaition $E$ }

 
\section{\textcolor{Chapter }{Decomposition}}\logpage{[ 2, 2, 0 ]}
\hyperdef{L}{X7911A60384C511AB}{}
{
 Let us now study equations over groups acting on a rooted tree without having
any explicitly given group in mind. Say $G\leq\operatorname{Aut}(\{1,2\}^*)$ and $g,h\in G$ and assume we want to see how the decomposition $\Phi_\gamma$ of the equation $E=[X,Y]g^Zh$ looks like. This decomposition will depend on the activity of $g$ and on $\gamma_{\textup{act}}$. 
\begin{Verbatim}[commandchars=!@|,fontsize=\small,frame=single,label=Example]
  !gapprompt@gap>| !gapinput@F := FreeGroup("X","Y","Z");; x:=F.1; y:=F.2; z:=F.3;|
  X
  Y
  Z
  !gapprompt@gap>| !gapinput@G := FreeGroup("g","h");; g:=G.1; h := G.2;|
  g
  h
  !gapprompt@gap>| !gapinput@S2 := [(),(1,2)];|
  !gapprompt@gap>| !gapinput@EqG := EquationGroup(G,F);;|
  !gapprompt@gap>| !gapinput@eq := Equation(EqG,[Comm(x,y)*z^-1,g,z,h]);|
  Equation in [ X, Y, Z ]
  !gapprompt@gap>| !gapinput@PhiE := [];|
  [ ]
  !gapprompt@gap>| !gapinput@for actg in S2 do|
  !gapprompt@>| !gapinput@      DeqG := DecompositionEquationGroup(EqG,2,[actg,()]);;|
  !gapprompt@>| !gapinput@      for gamma_act in Cartesian([S2,S2,S2]) do|
  !gapprompt@>| !gapinput@        Add(PhiE,DecompositionEquation(DeqG,eq,gamma_act));|
  !gapprompt@>| !gapinput@      od;|
  !gapprompt@>| !gapinput@   od;|
  !gapprompt@gap>| !gapinput@Print(PhiE[1]);|
  Equation([ FreeProductElm([ X1^-1*Y1^-1*X1*Y1*Z1^-1, g1, Z1, h1 ]), 
        FreeProductElm([ X2^-1*Y2^-1*X2*Y2*Z2^-1, g2, Z2, h2 ]) ]) 
  !gapprompt@gap>| !gapinput@Print(PhiE[16]);|
  Equation([ FreeProductElm([ X2^-1*Y1^-1*X1*Y2*Z2^-1, g2, Z1, h2 ]), 
        FreeProductElm([ X1^-1*Y2^-1*X2*Y1*Z1^-1, g1, Z2, h1 ]) ])       
   
\end{Verbatim}
 We see that for some (indeed for all but the first two cases) the states of
the decomposition do not form independent systems. Let us see how an
equivalent independent system looks like and find out which genus the
corresponding equations have: 
\begin{Verbatim}[commandchars=!@|,fontsize=\small,frame=single,label=Example]
  !gapprompt@gap>| !gapinput@Apply(PhiE,E->DecomposedEquationDisjointForm(E).eq);;|
  !gapprompt@gap>| !gapinput@Print(PhiE[16]);|
  Equation([ FreeProductElm([ X2^-1*Y1^-1*Y2^-1*X2*Y1*Z1^-1, g1, Z2, h1, Y2*Z2^-1,
   g2, Z1, h2 ]), FreeProductElm([  ]) ])
  !gapprompt@gap>| !gapinput@Genus(EquationComponent(PhiE[16],1));|
  2
  !gapprompt@gap>| !gapinput@List(PhiE,E->Genus(EquationComponent(E,1)));|
  [ 1, 1, 1, 1, 1, 1, 1, 1, 2, 2, 2, 2, 2, 2, 2, 2 ]
   
\end{Verbatim}
 }


\section{\textcolor{Chapter }{Using the fr package}}\logpage{[ 2, 3, 0 ]}
\hyperdef{L}{X85D8B17A7D594825}{}
{
 Finally let us do some computations in the Grigorchuk group. For example let
us compute a solution for the equation $E=[X,Y]dacab$. 
\begin{Verbatim}[commandchars=!@|,fontsize=\small,frame=single,label=Example]
  !gapprompt@gap>| !gapinput@LoadPackage("fr");;|
  !gapprompt@gap>| !gapinput@F := FreeGroup("X","Y");; SetName(F,"F"); x:=F.1;; y:=F.2;;|
  !gapprompt@gap>| !gapinput@G := GrigorchukGroup;;|
  !gapprompt@gap>| !gapinput@a:= G.1;; b:=G.2;; c:=G.3;; d:= G.4;;|
  !gapprompt@gap>| !gapinput@EqG := EquationGroup(G,F);;|
  !gapprompt@gap>| !gapinput@DeqG := DecompositionEquationGroup(EqG);|
  GrigorchukGroup*F*F
  !gapprompt@gap>| !gapinput@gamma_a := GroupHomomorphismByImages(F,SymmetricGroup(2),[(),(1,2)]);|
  [ X, Y ] -> [ (), (1,2) ]
  !gapprompt@gap>| !gapinput@eq := Equation(EqG,[Comm(x,y),d*a*c*a*b]);|
  Equation in [ X, Y ]
  !gapprompt@gap>| !gapinput@neq := DecompositionEquation(DeqG,eq,gamma_a);|
  DecomposedEquation in [ X1, X2, Y1, Y2 ]
  !gapprompt@gap>| !gapinput@deq := DecomposedEquationDisjointForm(neq);|
  rec( eq := DecomposedEquation in [ X1, X2, Y2 ], 
    hom := [ X1 ]"->"[ FreeProductElm of length 3 ] )
  !gapprompt@gap>| !gapinput@Nf := EquationNormalForm(EquationComponent(deq.eq,1));;|
  !gapprompt@gap>| !gapinput@F2 := FreeProductInfo(DeqG).groups[2];|
  F*F
  !gapprompt@gap>| !gapinput@s := EquationEvaluation(DeqG,[F2.1,F2.2,F2.3],[d,b,b]);|
  [ X1, X2, Y1 ]"->"[ d, b, b ]
  !gapprompt@gap>| !gapinput@IsSolution(s,Nf.nf);|
  true
  !gapprompt@gap>| !gapinput@IsSolution(Nf.hom*s,EquationComponent(deq.eq,1));|
  true
  !gapprompt@gap>| !gapinput@ForAll(EquationComponents(neq),E->IsSolution(deq.hom*Nf.hom*s,E));|
  true;
  !gapprompt@gap>| !gapinput@imgs := List(GeneratorsOfGroup(F2),x->ImageElm(deq.hom*Nf.hom*s,x));|
  [ <Mealy element on alphabet [ 1 .. 2 ] with 6 states>, 
    <Mealy element on alphabet [ 1 .. 2 ] with 7 states>, b^-1, 
    <Mealy element on alphabet [ 1 .. 2 ] with 9 states> ]
  !gapprompt@gap>| !gapinput@s2 := LiftSolution(neq,eq,gamma_a,deq.hom*Nf.hom*s);|
  [ X, Y ]"->"[ <Mealy element on alphabet [ 1 .. 2 ] with 9 states>, 
    <Mealy element on alphabet [ 1 .. 2 ] with 10 states> ]
  !gapprompt@gap>| !gapinput@IsSolution(s2,eq);|
  true;
   
\end{Verbatim}
}

 }

 
\chapter{\textcolor{Chapter }{FreeProducts}}\logpage{[ 3, 0, 0 ]}
\hyperdef{L}{X7C7E004E7A1F3ECE}{}
{
 
\section{\textcolor{Chapter }{Construction}}\logpage{[ 3, 1, 0 ]}
\hyperdef{L}{X7F6278CD87400D49}{}
{
 This package installs some new method for the command \texttt{FreeProduct}. Before it was only possible to construct free products of finitely presented
groups. 

 If the resulting group was constructed by the new methods they will be in the
following filter: \texttt{IsGeneralFreeProduct} }

 
\section{\textcolor{Chapter }{Filters}}\logpage{[ 3, 2, 0 ]}
\hyperdef{L}{X84EFA4C07D4277BB}{}
{
 

\subsection{\textcolor{Chapter }{IsGeneralFreeProduct}}
\logpage{[ 3, 2, 1 ]}\nobreak
\hyperdef{L}{X7B0634527B667F02}{}
{\noindent\textcolor{FuncColor}{$\triangleright$\ \ \texttt{IsGeneralFreeProduct({\mdseries\slshape obj})\index{IsGeneralFreeProduct@\texttt{IsGeneralFreeProduct}}
\label{IsGeneralFreeProduct}
}\hfill{\scriptsize (filter)}}\\
\noindent\textcolor{FuncColor}{$\triangleright$\ \ \texttt{IsFreeProductElm({\mdseries\slshape obj})\index{IsFreeProductElm@\texttt{IsFreeProductElm}}
\label{IsFreeProductElm}
}\hfill{\scriptsize (filter)}}\\
\noindent\textcolor{FuncColor}{$\triangleright$\ \ \texttt{IsFreeProductHomomorphism({\mdseries\slshape obj})\index{IsFreeProductHomomorphism@\texttt{IsFreeProductHomomorphism}}
\label{IsFreeProductHomomorphism}
}\hfill{\scriptsize (filter)}}\\
\textbf{\indent Returns:\ }
\texttt{true} if \mbox{\texttt{\mdseries\slshape obj}} is a general free product,a free product element, a free product homomorphism.



 These filters can be used to check weather a given group was created as
general free product etc. }

 }

 
\section{\textcolor{Chapter }{Construction}}\logpage{[ 3, 3, 0 ]}
\hyperdef{L}{X7F6278CD87400D49}{}
{
 

\subsection{\textcolor{Chapter }{GeneralFreeProduct (group)}}
\logpage{[ 3, 3, 1 ]}\nobreak
\hyperdef{L}{X7D290915832D6A52}{}
{\noindent\textcolor{FuncColor}{$\triangleright$\ \ \texttt{GeneralFreeProduct({\mdseries\slshape group})\index{GeneralFreeProduct@\texttt{GeneralFreeProduct}!group}
\label{GeneralFreeProduct:group}
}\hfill{\scriptsize (operation)}}\\
\textbf{\indent Returns:\ }
A a new general free product isomorphic to \mbox{\texttt{\mdseries\slshape group}}.



 Takes a group which has free product information stored and returns a new
group which lies in the filter \texttt{IsGeneralFreeProduct}. The returned groups represents the free product of the groups in \texttt{FreeProductInfo.groups}. 
\begin{Verbatim}[commandchars=!@|,fontsize=\small,frame=single,label=Example]
  !gapprompt@gap>| !gapinput@S2 := SymmetricGroup(2);; SetName(S2,"S2");|
  !gapprompt@gap>| !gapinput@S3 := SymmetricGroup(3);; SetName(F2,"F2");|
  !gapprompt@gap>| !gapinput@G := FreeProduct(S2,S3);|
  <fp group on the generators [ f1, f2, f3 ]>
  !gapprompt@gap>| !gapinput@G := GeneralFreeProduct(G);|
  S2*S3
\end{Verbatim}
 }

 

\subsection{\textcolor{Chapter }{GeneratorsOfGroup (group)}}
\logpage{[ 3, 3, 2 ]}\nobreak
\hyperdef{L}{X7E10402B7BF28A13}{}
{\noindent\textcolor{FuncColor}{$\triangleright$\ \ \texttt{GeneratorsOfGroup({\mdseries\slshape group})\index{GeneratorsOfGroup@\texttt{GeneratorsOfGroup}!group}
\label{GeneratorsOfGroup:group}
}\hfill{\scriptsize (operation)}}\\
\textbf{\indent Returns:\ }
The generators of \mbox{\texttt{\mdseries\slshape group}}.

}

 

\subsection{\textcolor{Chapter }{\texttt{\symbol{92}}= (G,H)}}
\logpage{[ 3, 3, 3 ]}\nobreak
\hyperdef{L}{X81EA40587F556E04}{}
{\noindent\textcolor{FuncColor}{$\triangleright$\ \ \texttt{\texttt{\symbol{92}}=({\mdseries\slshape group, group})\index{=@\texttt{\texttt{\symbol{92}}=}!G,H}
\label{=:G,H}
}\hfill{\scriptsize (operation)}}\\
\textbf{\indent Returns:\ }
True if the free factors of the groups \mbox{\texttt{\mdseries\slshape G}} and \mbox{\texttt{\mdseries\slshape H}} are equal.

}

 }

 
\section{\textcolor{Chapter }{Elements}}\logpage{[ 3, 4, 0 ]}
\hyperdef{L}{X79B130FC7906FB4C}{}
{
 

\subsection{\textcolor{Chapter }{FreeProductElm (group,list,list)}}
\logpage{[ 3, 4, 1 ]}\nobreak
\hyperdef{L}{X8302A20E7C57546D}{}
{\noindent\textcolor{FuncColor}{$\triangleright$\ \ \texttt{FreeProductElm({\mdseries\slshape group, word, factors})\index{FreeProductElm@\texttt{FreeProductElm}!group,list,list}
\label{FreeProductElm:group,list,list}
}\hfill{\scriptsize (operation)}}\\
\noindent\textcolor{FuncColor}{$\triangleright$\ \ \texttt{FreeProductElmLetterRep({\mdseries\slshape group, word, factors})\index{FreeProductElmLetterRep@\texttt{FreeProductElmLetterRep}!group,list,list}
\label{FreeProductElmLetterRep:group,list,list}
}\hfill{\scriptsize (operation)}}\\
\textbf{\indent Returns:\ }
A new element in the group \mbox{\texttt{\mdseries\slshape group}}.



 This function constructs a new free product element, belonging to the group \mbox{\texttt{\mdseries\slshape group}}. 

 \mbox{\texttt{\mdseries\slshape words}} is a dense list of elements of any of the factors of \mbox{\texttt{\mdseries\slshape group}}. 

 \mbox{\texttt{\mdseries\slshape factors}} is a list of integers. \mbox{\texttt{\mdseries\slshape word}}[i] must lie in the factor \mbox{\texttt{\mdseries\slshape factors}}[\mbox{\texttt{\mdseries\slshape i}}] of \mbox{\texttt{\mdseries\slshape group}}. If this is not the case an error is thrown. 

 \texttt{FreeProductElmLetterRep} does not simplify the word by multipliying neighbored equal factor elements
but stores the letters as given. 
\begin{Verbatim}[commandchars=!@|,fontsize=\small,frame=single,label=Example]
  !gapprompt@gap>| !gapinput@F2 := FreeGroup(2);; SetName(F2,"F2");|
  !gapprompt@gap>| !gapinput@S4 := SymmetricGroup(4);; SetName(S4,"S4");|
  !gapprompt@gap>| !gapinput@G := FreeProduct(F2,S4);|
  F2*S4
  !gapprompt@gap>| !gapinput@e := FreeProductElm(G,[F2.1,F2.2,(1,2),F2.1],[1,1,2,1]);|
  FreeProductElm of length 3
  !gapprompt@gap>| !gapinput@Print(e^2);|
  FreeProductElm([ f1*f2, (1,2), f1^2*f2, (1,2), f1 ])
  !gapprompt@gap>| !gapinput@Print(FreeProductElmLetterRep(G,[F2.1,F2.2,(1,2),F2.1],[1,1,2,1]));|
  FreeProductElm([ f1, f2, (1,2), f1 ])
\end{Verbatim}
 }

 There are two representations for this kind of elements. 

\subsection{\textcolor{Chapter }{IsFreeProductElmRep}}
\logpage{[ 3, 4, 2 ]}\nobreak
\hyperdef{L}{X7E42E9018734D9F2}{}
{\noindent\textcolor{FuncColor}{$\triangleright$\ \ \texttt{IsFreeProductElmRep({\mdseries\slshape obj})\index{IsFreeProductElmRep@\texttt{IsFreeProductElmRep}}
\label{IsFreeProductElmRep}
}\hfill{\scriptsize (filter)}}\\
\noindent\textcolor{FuncColor}{$\triangleright$\ \ \texttt{IsFreeProductElmLetterRep({\mdseries\slshape obj})\index{IsFreeProductElmLetterRep@\texttt{IsFreeProductElmLetterRep}}
\label{IsFreeProductElmLetterRep}
}\hfill{\scriptsize (filter)}}\\
\textbf{\indent Returns:\ }
\texttt{true} if \mbox{\texttt{\mdseries\slshape obj}} is a general free product element in standard/letter storing representation.



 }

 }

 
\section{\textcolor{Chapter }{Basic operations}}\logpage{[ 3, 5, 0 ]}
\hyperdef{L}{X82EB5BE77F9F686A}{}
{
 

\subsection{\textcolor{Chapter }{\texttt{\symbol{92}}* (freeproductelm,freeproductelm)}}
\logpage{[ 3, 5, 1 ]}\nobreak
\hyperdef{L}{X819C33F77AA12DF3}{}
{\noindent\textcolor{FuncColor}{$\triangleright$\ \ \texttt{\texttt{\symbol{92}}*({\mdseries\slshape e1, e2})\index{*@\texttt{\texttt{\symbol{92}}*}!freeproductelm,freeproductelm}
\label{*:freeproductelm,freeproductelm}
}\hfill{\scriptsize (operation)}}\\
\textbf{\indent Returns:\ }
The product of the two elements.

}

 

\subsection{\textcolor{Chapter }{\texttt{\symbol{92}}* (freeproductelm)}}
\logpage{[ 3, 5, 2 ]}\nobreak
\hyperdef{L}{X7B9E235D8600EA8F}{}
{\noindent\textcolor{FuncColor}{$\triangleright$\ \ \texttt{\texttt{\symbol{92}}*({\mdseries\slshape elm})\index{*@\texttt{\texttt{\symbol{92}}*}!freeproductelm}
\label{*:freeproductelm}
}\hfill{\scriptsize (operation)}}\\
\textbf{\indent Returns:\ }
The inverse element

}

 

\subsection{\textcolor{Chapter }{OneOp (freeproductelm)}}
\logpage{[ 3, 5, 3 ]}\nobreak
\hyperdef{L}{X7D3AC29E851BC9D3}{}
{\noindent\textcolor{FuncColor}{$\triangleright$\ \ \texttt{OneOp({\mdseries\slshape elm})\index{OneOp@\texttt{OneOp}!freeproductelm}
\label{OneOp:freeproductelm}
}\hfill{\scriptsize (operation)}}\\
\textbf{\indent Returns:\ }
The identity element

}

 

\subsection{\textcolor{Chapter }{\texttt{\symbol{92}}= (freeproductelm,freeproductelm)}}
\logpage{[ 3, 5, 4 ]}\nobreak
\hyperdef{L}{X7DF0A285829C15AF}{}
{\noindent\textcolor{FuncColor}{$\triangleright$\ \ \texttt{\texttt{\symbol{92}}=({\mdseries\slshape e1, ee2})\index{=@\texttt{\texttt{\symbol{92}}=}!freeproductelm,freeproductelm}
\label{=:freeproductelm,freeproductelm}
}\hfill{\scriptsize (operation)}}\\
\textbf{\indent Returns:\ }
True if the two elements are equal.

}

 \# 

\subsection{\textcolor{Chapter }{Length (freeproductelm)}}
\logpage{[ 3, 5, 5 ]}\nobreak
\hyperdef{L}{X863F4521823BC021}{}
{\noindent\textcolor{FuncColor}{$\triangleright$\ \ \texttt{Length({\mdseries\slshape e1})\index{Length@\texttt{Length}!freeproductelm}
\label{Length:freeproductelm}
}\hfill{\scriptsize (operation)}}\\
\textbf{\indent Returns:\ }
The length of the list that stores the elements of the free factors

}

 

\subsection{\textcolor{Chapter }{\texttt{\symbol{92}}[\texttt{\symbol{92}}] (freeproductelm,integer)}}
\logpage{[ 3, 5, 6 ]}\nobreak
\hyperdef{L}{X7A12911186A5262D}{}
{\noindent\textcolor{FuncColor}{$\triangleright$\ \ \texttt{\texttt{\symbol{92}}[\texttt{\symbol{92}}]({\mdseries\slshape e1, i})\index{[]@\texttt{\texttt{\symbol{92}}[\texttt{\symbol{92}}]}!freeproductelm,integer}
\label{[]:freeproductelm,integer}
}\hfill{\scriptsize (operation)}}\\
\textbf{\indent Returns:\ }
The free product element consisting only of the \mbox{\texttt{\mdseries\slshape i}}-th entry of the underlying list of elements.

}

 

\subsection{\textcolor{Chapter }{Position (freeproductelm)}}
\logpage{[ 3, 5, 7 ]}\nobreak
\hyperdef{L}{X7F19F8137D3C1714}{}
{\noindent\textcolor{FuncColor}{$\triangleright$\ \ \texttt{Position({\mdseries\slshape e1})\index{Position@\texttt{Position}!freeproductelm}
\label{Position:freeproductelm}
}\hfill{\scriptsize (operation)}}\\
\textbf{\indent Returns:\ }
The position of the element \mbox{\texttt{\mdseries\slshape el}} in the underlyig list.



 
\begin{Verbatim}[commandchars=!@|,fontsize=\small,frame=single,label=Example]
  !gapprompt@gap>| !gapinput@F2 := FreeGroup(2);; SetName(F2,"F2");|
  !gapprompt@gap>| !gapinput@S4 := SymmetricGroup(4);; SetName(S4,"S4");|
  !gapprompt@gap>| !gapinput@G := FreeProduct(F2,S4);|
  F2*S4
  !gapprompt@gap>| !gapinput@e := FreeProductElm(G,[F2.1,F2.2,(1,2),F2.1],[1,1,2,1]);;Print(e);|
  FreeProductElm([ f1*f2, (1,2), f1 ])
  !gapprompt@gap>| !gapinput@Length(e);|
  3
  !gapprompt@gap>| !gapinput@Position(e,(1,2));|
  2
  !gapprompt@gap>| !gapinput@Print(e[1]);|
  FreeProductElm([ f1*f2 ])
\end{Verbatim}
 }

 }

 
\section{\textcolor{Chapter }{Homomorphisms}}\logpage{[ 3, 6, 0 ]}
\hyperdef{L}{X84975388859F203D}{}
{
 

\subsection{\textcolor{Chapter }{FreeProductHomomorphism (group,group,list)}}
\logpage{[ 3, 6, 1 ]}\nobreak
\hyperdef{L}{X79123B6B830D92FE}{}
{\noindent\textcolor{FuncColor}{$\triangleright$\ \ \texttt{FreeProductHomomorphism({\mdseries\slshape source, target, homs})\index{FreeProductHomomorphism@\texttt{FreeProductHomomorphism}!group,group,list}
\label{FreeProductHomomorphism:group,group,list}
}\hfill{\scriptsize (operation)}}\\
\textbf{\indent Returns:\ }
A new group homomorphism from \mbox{\texttt{\mdseries\slshape source}} to \mbox{\texttt{\mdseries\slshape target}}.



 This function constructs a new group homomorphism from the general free
product group \mbox{\texttt{\mdseries\slshape source}} to the general free product group \mbox{\texttt{\mdseries\slshape target}} by mapping the factor \texttt{i} by the group homomorphism \mbox{\texttt{\mdseries\slshape homs}}[\texttt{i}] to the \texttt{i}th factor of \mbox{\texttt{\mdseries\slshape target}}. 

 \mbox{\texttt{\mdseries\slshape homs}} is a dense list of group homomorphisms where the source of \mbox{\texttt{\mdseries\slshape homs}}[\texttt{i}] must be the \texttt{i}th factor of \mbox{\texttt{\mdseries\slshape source}} and the range of \mbox{\texttt{\mdseries\slshape homs}}[\texttt{i}] must be the \texttt{i}th factor of \mbox{\texttt{\mdseries\slshape target}}. 
\begin{Verbatim}[commandchars=!@|,fontsize=\small,frame=single,label=Example]
  !gapprompt@gap>| !gapinput@F2 := FreeGroup(2);; SetName(F2,"F2");|
  !gapprompt@gap>| !gapinput@S4 := SymmetricGroup(4);; SetName(S4,"S4");|
  !gapprompt@gap>| !gapinput@A4 := AlternatingGroup(4);; SetName(A4,"A4");|
  !gapprompt@gap>| !gapinput@G := FreeProduct(F2,S4); H := FreeProduct(F2,A4);|
  F2*S4
  F2*A4
  !gapprompt@gap>| !gapinput@hf := GroupHomomorphismByImages(F2,F2,[F2.2,F2.1]);;|
  !gapprompt@gap>| !gapinput@hg := GroupHomomorphismByFunction(S4,A4,s->Comm(s,S4.2));;|
  !gapprompt@gap>| !gapinput@h := FreeProductHomomorphism(G,H,[hf,hg]);|
  <mapping: F2*S4 -> F2*A4 >
  !gapprompt@gap>| !gapinput@e := FreeProductElm(G,[F2.1,F2.2,(1,2),F2.1],[1,1,2,1]);|
  FreeProductElm of length 3
  !gapprompt@gap>| !gapinput@Print(e^h);|
  FreeProductElm([ f2*f1*f2 ])
\end{Verbatim}
 }

 

\subsection{\textcolor{Chapter }{IsGeneralFreeProductRep}}
\logpage{[ 3, 6, 2 ]}\nobreak
\hyperdef{L}{X87011CEA7ACB5387}{}
{\noindent\textcolor{FuncColor}{$\triangleright$\ \ \texttt{IsGeneralFreeProductRep({\mdseries\slshape obj})\index{IsGeneralFreeProductRep@\texttt{IsGeneralFreeProductRep}}
\label{IsGeneralFreeProductRep}
}\hfill{\scriptsize (filter)}}\\
\textbf{\indent Returns:\ }
\texttt{true} if \mbox{\texttt{\mdseries\slshape obj}} is a general free product element in standard/letter storing representation.



 }

 }

 
\section{\textcolor{Chapter }{Other operations}}\logpage{[ 3, 7, 0 ]}
\hyperdef{L}{X87497C207B7D7511}{}
{
  

\subsection{\textcolor{Chapter }{Abs (assocword)}}
\logpage{[ 3, 7, 1 ]}\nobreak
\hyperdef{L}{X7C60A2C185A61AF7}{}
{\noindent\textcolor{FuncColor}{$\triangleright$\ \ \texttt{Abs({\mdseries\slshape obj})\index{Abs@\texttt{Abs}!assocword}
\label{Abs:assocword}
}\hfill{\scriptsize (operation)}}\\
\textbf{\indent Returns:\ }
An assocword without inverses of generators



 In the word \mbox{\texttt{\mdseries\slshape obj}} all occurencies of inverse generators are replaced by the coresponding
generators. 
\begin{Verbatim}[commandchars=!@|,fontsize=\small,frame=single,label=Example]
  !gapprompt@gap>| !gapinput@F2 := FreeGroup(2);; SetName(F2,"F2");|
  !gapprompt@gap>| !gapinput@w := F2.1^-1*F2.2*F2.1*F2.2^-1;|
  f1^-1*f2*f1*f2^-1
  !gapprompt@gap>| !gapinput@Abs(w);|
  (f1*f2)^2
\end{Verbatim}
 }

 }

 }

 
\chapter{\textcolor{Chapter }{Equations}}\logpage{[ 4, 0, 0 ]}
\hyperdef{L}{X7FA73DBF8424437E}{}
{
 We fix a set $\mathcal{X}$ and call its elements \emph{variables}. We assume that $\mathcal{X}$ is infinite countable, is well ordered, and its family of finite subsets is
also well ordered, by size and then lexicographic order. We denote by $F_\mathcal{X}$ the free group on the generating set $\mathcal{X}$. 

 Let $G$ be a group. The \emph{equation group} will be the free product $G*F_\mathcal{X}$ and the elements belonging to $G$ will be called \emph{constants}. 

 A $G$-equation is an element $E$ of the group $F_\mathcal{X}*G$ regarded as a reduced word. For $E$ a $G$-equation, its set of \emph{variables} $\textup{Var}(E)\subset \mathcal{X}$ is the set of symbols in $\mathcal{X}$ that occur in it; namely, $\textup{Var}(E)$ is the minimal subset of $\mathcal{X}$ such that $E$ belongs to $F_{\textup{Var}(E)}*G$. 

 A \emph{quadratic} equation is an equation in which each variable $X\in\textup{Var}(E)$ occurs exactly twice. A quadratic equation is called \emph{oriented} if for each variable $X\in\textup{Var(\mathcal{X})}$ both letters $X$ and $X^{-1}$ occure in the reduced word $E$. 
\section{\textcolor{Chapter }{Construction}}\logpage{[ 4, 1, 0 ]}
\hyperdef{L}{X7F6278CD87400D49}{}
{
 

\subsection{\textcolor{Chapter }{IsEquationGroup}}
\logpage{[ 4, 1, 1 ]}\nobreak
\hyperdef{L}{X808D75B58057C4D4}{}
{\noindent\textcolor{FuncColor}{$\triangleright$\ \ \texttt{IsEquationGroup({\mdseries\slshape obj})\index{IsEquationGroup@\texttt{IsEquationGroup}}
\label{IsEquationGroup}
}\hfill{\scriptsize (filter)}}\\
\textbf{\indent Returns:\ }
\texttt{true} if \mbox{\texttt{\mdseries\slshape obj}} is a general free product over to groups \texttt{G,F} where \texttt{F} is a free group.



 The free factor \texttt{F} represents the group of variables for the equations. }



\subsection{\textcolor{Chapter }{EquationGroup (group,group)}}
\logpage{[ 4, 1, 2 ]}\nobreak
\hyperdef{L}{X7EF6B973848879A1}{}
{\noindent\textcolor{FuncColor}{$\triangleright$\ \ \texttt{EquationGroup({\mdseries\slshape G, F})\index{EquationGroup@\texttt{EquationGroup}!group,group}
\label{EquationGroup:group,group}
}\hfill{\scriptsize (operation)}}\\
\textbf{\indent Returns:\ }
A a new \mbox{\texttt{\mdseries\slshape G}}-group for equations over \mbox{\texttt{\mdseries\slshape G}}.



 Uses the \texttt{FreeProduct} method to create the free product object. The second argument \mbox{\texttt{\mdseries\slshape F}} must be a free group. 
\begin{Verbatim}[commandchars=!@|,fontsize=\small,frame=single,label=Example]
  !gapprompt@gap>| !gapinput@S2 := SymmetricGroup(2);; SetName(S2,"S2");|
  !gapprompt@gap>| !gapinput@F := FreeGroup(infinity,"xn",["x1","x2"]);;SetName(F,"F");|
  !gapprompt@gap>| !gapinput@EqG := EquationGroup(S2,F);|
  S2*F
\end{Verbatim}
 }

 

\subsection{\textcolor{Chapter }{Equation (group,list)}}
\logpage{[ 4, 1, 3 ]}\nobreak
\hyperdef{L}{X78102FC383761659}{}
{\noindent\textcolor{FuncColor}{$\triangleright$\ \ \texttt{Equation({\mdseries\slshape G, L})\index{Equation@\texttt{Equation}!group,list}
\label{Equation:group,list}
}\hfill{\scriptsize (operation)}}\\
\textbf{\indent Returns:\ }
A a new element of the equation group \mbox{\texttt{\mdseries\slshape G}}



 Creates a \texttt{FreeProductElm} from the list \mbox{\texttt{\mdseries\slshape L}}. By default this elements will be cyclical reduced. }

 

\subsection{\textcolor{Chapter }{EquationVariables (groupelement)}}
\logpage{[ 4, 1, 4 ]}\nobreak
\hyperdef{L}{X806BB78E7A8B9FB1}{}
{\noindent\textcolor{FuncColor}{$\triangleright$\ \ \texttt{EquationVariables({\mdseries\slshape E})\index{EquationVariables@\texttt{EquationVariables}!groupelement}
\label{EquationVariables:groupelement}
}\hfill{\scriptsize (attribute)}}\\
\textbf{\indent Returns:\ }
A list of all variables occuring in \mbox{\texttt{\mdseries\slshape E}}.

}

 

\subsection{\textcolor{Chapter }{EquationLetterRep (equation)}}
\logpage{[ 4, 1, 5 ]}\nobreak
\hyperdef{L}{X82B2E4C181B54921}{}
{\noindent\textcolor{FuncColor}{$\triangleright$\ \ \texttt{EquationLetterRep({\mdseries\slshape E})\index{EquationLetterRep@\texttt{EquationLetterRep}!equation}
\label{EquationLetterRep:equation}
}\hfill{\scriptsize (attribute)}}\\
\textbf{\indent Returns:\ }
A a new element of the equation group \mbox{\texttt{\mdseries\slshape G}} in letter representation which is equal to \mbox{\texttt{\mdseries\slshape E}}



 In the standard representation of an equation the elements of the free group
that are not devided by a constant are collected. In the letter representation
they are seperate letters. 
\begin{Verbatim}[commandchars=!@|,fontsize=\small,frame=single,label=Example]
  !gapprompt@gap>| !gapinput@F2 := FreeGroup(2);; SetName(F2,"F2");|
  !gapprompt@gap>| !gapinput@S4 := SymmetricGroup(4);; SetName(S4,"S4");|
  !gapprompt@gap>| !gapinput@G := EquationGroup(S4,F2);|
  S4*F2
  !gapprompt@gap>| !gapinput@e := Equation(G,[F2.1,F2.2,(1,2),F2.1]);|
  Equation in [ f1, f2 ]
  !gapprompt@gap>| !gapinput@Print(e);|
  FreeProductElm([ f1*f2, (1,2), f1 ])
  !gapprompt@gap>| !gapinput@Print(EquationLetterRep(e));|
  FreeProductElm([ f1, f2, (1,2), f1 ])
\end{Verbatim}
 }

 

\subsection{\textcolor{Chapter }{EquationLetterRep (group,list)}}
\logpage{[ 4, 1, 6 ]}\nobreak
\hyperdef{L}{X798A4D65860BFA88}{}
{\noindent\textcolor{FuncColor}{$\triangleright$\ \ \texttt{EquationLetterRep({\mdseries\slshape G, L})\index{EquationLetterRep@\texttt{EquationLetterRep}!group,list}
\label{EquationLetterRep:group,list}
}\hfill{\scriptsize (attribute)}}\\
\textbf{\indent Returns:\ }
Creates a new equation in letter representation

}

 

\subsection{\textcolor{Chapter }{IsQuadraticEquation (equation)}}
\logpage{[ 4, 1, 7 ]}\nobreak
\hyperdef{L}{X86E2E96679F5A416}{}
{\noindent\textcolor{FuncColor}{$\triangleright$\ \ \texttt{IsQuadraticEquation({\mdseries\slshape E})\index{IsQuadraticEquation@\texttt{IsQuadraticEquation}!equation}
\label{IsQuadraticEquation:equation}
}\hfill{\scriptsize (property)}}\\
\textbf{\indent Returns:\ }
\texttt{true} if \mbox{\texttt{\mdseries\slshape E}} is an quadratic equation.



 }

 

\subsection{\textcolor{Chapter }{IsOrientedEquation (equation)}}
\logpage{[ 4, 1, 8 ]}\nobreak
\hyperdef{L}{X818F7C7186D0FE91}{}
{\noindent\textcolor{FuncColor}{$\triangleright$\ \ \texttt{IsOrientedEquation({\mdseries\slshape E})\index{IsOrientedEquation@\texttt{IsOrientedEquation}!equation}
\label{IsOrientedEquation:equation}
}\hfill{\scriptsize (property)}}\\
\textbf{\indent Returns:\ }
\texttt{true} if \mbox{\texttt{\mdseries\slshape E}} is an oriented quadratic equation.



 }

 }


\section{\textcolor{Chapter }{Homomorphisms}}\logpage{[ 4, 2, 0 ]}
\hyperdef{L}{X84975388859F203D}{}
{
 An \emph{evaluation} is a $G$-homomorphism $e\colon F_{\mathcal{X}} * G \to G$. A \emph{solution} of an equation $E$ is an evaluation $s$ satisfying $s(E)=1$. If a solution exists for $E$ then the equation $E$ is called \emph{solvable}. The set of elements $X\in \mathcal{X}$ with $s(X)\neq 1$ is called the \emph{support} of the solution. 

\subsection{\textcolor{Chapter }{EquationHomomorphism (group,list,list)}}
\logpage{[ 4, 2, 1 ]}\nobreak
\hyperdef{L}{X7B433C8879AECD75}{}
{\noindent\textcolor{FuncColor}{$\triangleright$\ \ \texttt{EquationHomomorphism({\mdseries\slshape G, vars, imgs})\index{EquationHomomorphism@\texttt{EquationHomomorphism}!group,list,list}
\label{EquationHomomorphism:group,list,list}
}\hfill{\scriptsize (operation)}}\\
\textbf{\indent Returns:\ }
A a new homomorphism from \mbox{\texttt{\mdseries\slshape G}} to \mbox{\texttt{\mdseries\slshape G}}



 If \mbox{\texttt{\mdseries\slshape G}} is the group $H*F_X$ the result of this command is a $H$-homomorphism that maps the $i$-th variable of the list \mbox{\texttt{\mdseries\slshape vars}} to the $i$-th member of \mbox{\texttt{\mdseries\slshape imgs}}. Therefore \mbox{\texttt{\mdseries\slshape vars}} can be a list without duplicates of variables. The list \mbox{\texttt{\mdseries\slshape imgs}} can contain elements of the following type: 
\begin{itemize}
\item Element of the group $F_X$
\item Elements of the group $H$
\item Lists of elements from the groups $F_X$ and $H$. The list is then regarded as the corresponding word in $G$
\item Elements of the group $G$
\end{itemize}
 
\begin{Verbatim}[commandchars=!@|,fontsize=\small,frame=single,label=Example]
  !gapprompt@gap>| !gapinput@F3 := FreeGroup(3);; SetName(F3,"F3");|
  !gapprompt@gap>| !gapinput@S4 := SymmetricGroup(4);; SetName(S4,"S4");|
  !gapprompt@gap>| !gapinput@G := EquationGroup(S4,F3);|
  S4*F3
  !gapprompt@gap>| !gapinput@e := Equation(G,[Comm(F3.2,F3.1)*F3.3^2,(1,2)]);|
  Equation in [ f1, f2, f3 ]
  !gapprompt@gap>| !gapinput@ h := EquationHomomorphism(G,[F3.1,F3.2,F3.3],|
  !gapprompt@>| !gapinput@[F3.1*F3.2*F3.3,(F3.2*F3.3)^(F3.1*F3.2*F3.3),(F3.2^-1*F3.1^-1)^F3.3]);|
  [ f1, f2, f3 ]"->"[ f1*f2*f3, f3^-1*f2^-1*f1^-1*f2*f3*f1*f2*f3, f3^-1*f2^-1*f1^-1*f3 ]
  !gapprompt@gap>| !gapinput@Print(e^h);|
  FreeProductElm([ f1^2*f2^2*f3^2, (1,2) ])
\end{Verbatim}
}

 

\subsection{\textcolor{Chapter }{EquationEvaluation (group,list,list)}}
\logpage{[ 4, 2, 2 ]}\nobreak
\hyperdef{L}{X804EE0107B532B5F}{}
{\noindent\textcolor{FuncColor}{$\triangleright$\ \ \texttt{EquationEvaluation({\mdseries\slshape G, vars, imgs})\index{EquationEvaluation@\texttt{EquationEvaluation}!group,list,list}
\label{EquationEvaluation:group,list,list}
}\hfill{\scriptsize (operation)}}\\
\textbf{\indent Returns:\ }
A a new evaluation from \mbox{\texttt{\mdseries\slshape G}}



 Works the same as \emph{EquationHomomorphism} but the target of the homomorphism is the group of constants and all variables
which are not specified in in \mbox{\texttt{\mdseries\slshape vars}} are maped to the identity. Hence the only allowed input for \mbox{\texttt{\mdseries\slshape imgs}} are elements of the group of constants. 
\begin{Verbatim}[commandchars=!@|,fontsize=\small,frame=single,label=Example]
  !gapprompt@gap>| !gapinput@F3 := FreeGroup(3);; SetName(F3,"F3");|
  !gapprompt@gap>| !gapinput@S4 := SymmetricGroup(4);; SetName(S4,"S4");|
  !gapprompt@gap>| !gapinput@G := EquationGroup(S4,F3);|
  S4*F3
  !gapprompt@gap>| !gapinput@e := Equation(G,[Comm(F3.2,F3.1)*F3.3^2,(1,2,3)]);|
  Equation in [ f1, f2, f3 ]
  !gapprompt@gap>| !gapinput@ h := EquationHomomorphism(G,[F3.1,F3.2,F3.3],[(),(),(1,2,3)]);|
  [ f1, f2, f3 ]"->"[ (), (), (1,3,2) ]
  !gapprompt@gap>| !gapinput@ he := EquationEvaluation(G,[F3.1,F3.2,F3.3],[(),(),(1,2,3)]);|
  MappingByFunction( S4*F3, S4, function( q ) ... end )
  !gapprompt@gap>| !gapinput@e^he;|
  ()
  !gapprompt@gap>| !gapinput@IsSolution(he,e);|
  true
\end{Verbatim}
 }

 }

 
\section{\textcolor{Chapter }{Normal Form}}\logpage{[ 4, 3, 0 ]}
\hyperdef{L}{X7EAA7DBE783858B7}{}
{
 For $m,n\ge0$, $X_i,Y_i,Z_i \in \mathcal{X}$ and $c_i \in G$ the following two kinds of equations are called in \emph{normal form}: \begin{center}
\begin{tabular}{rl}$O_{n,m}:$&
$[X_1,Y_1][X_2,Y_2]\cdots[X_n,Y_n]c_1^{Z_1}\cdots c_{m-1}^{Z_{m-1}}c_m$\\
$U_{n,m}:$&
$X_1^2X_2^2\cdots X_n^2 c_1^{Z_1}\cdots c_{m-1}^{Z_{m-1}}c_m\ .$\\
\end{tabular}\\[2mm]
\end{center}

 

 The form $O_{n,m}$ is called the oriented case and $U_{n,m}$ for $n>0$ the unoriented case. The parameter $n$ is referred to as \emph{genus} of the normal form of an equation. The pair $(n,m)$ will be called the \emph{signature} of the quadratic equation. It was proven by Commerford and Edmunds (\cite{Comerford-Edmunds:EquationsFreeGroups}) that every quadratic equation is isomorphic to one of the form $O_{n,m}$ or $U_{n,m}$ by an $G$-isomorphism. 

\subsection{\textcolor{Chapter }{EquationNormalForm (equation)}}
\logpage{[ 4, 3, 1 ]}\nobreak
\hyperdef{L}{X7D74F2867C61EA44}{}
{\noindent\textcolor{FuncColor}{$\triangleright$\ \ \texttt{EquationNormalForm({\mdseries\slshape E})\index{EquationNormalForm@\texttt{EquationNormalForm}!equation}
\label{EquationNormalForm:equation}
}\hfill{\scriptsize (operation)}}\\
\textbf{\indent Returns:\ }
A record with two $3$ components: \mbox{\texttt{\mdseries\slshape nf}}, \mbox{\texttt{\mdseries\slshape hom}} and \mbox{\texttt{\mdseries\slshape homInv}}



 The argument \mbox{\texttt{\mdseries\slshape E}} needs to be a quadratic equation. For each such equation there exists an
equivalent equation in normal form.

 The component \mbox{\texttt{\mdseries\slshape nf}} is an equation in one of the forms $O_{n,m},U_{n,m}$ equivalent to the equation \mbox{\texttt{\mdseries\slshape E}}. The component \mbox{\texttt{\mdseries\slshape hom}} is an equation homomorphism which maps \mbox{\texttt{\mdseries\slshape E}} to \mbox{\texttt{\mdseries\slshape nf}}. The component \mbox{\texttt{\mdseries\slshape homInv}} is the inverse of this homomorphism. 
\begin{Verbatim}[commandchars=!@|,fontsize=\small,frame=single,label=Example]
  !gapprompt@gap>| !gapinput@F3 := FreeGroup("x","y","z");; SetName(F3,"F3");|
  !gapprompt@gap>| !gapinput@S4 := SymmetricGroup(4);; SetName(S4,"S4");|
  !gapprompt@gap>| !gapinput@G := EquationGroup(S4,F3);|
  S4*F3
  !gapprompt@gap>| !gapinput@e := Equation(G,[Comm(F3.2,F3.1)*F3.3^2,(1,2)]);|
  Equation in [ x, y, z ]
  !gapprompt@gap>| !gapinput@ nf := EquationNormalForm(e);;|
  !gapprompt@gap>| !gapinput@Print(nf.nf);|
  FreeProductElm([ x^2*y^2*z^2, (1,2) ])
  !gapprompt@gap>| !gapinput@e^(nf.hom)=nf.nf;|
  true
  !gapprompt@gap>| !gapinput@nf.nf^(nf.homInv)=e;|
  true
\end{Verbatim}
}

 

\subsection{\textcolor{Chapter }{Genus (equation)}}
\logpage{[ 4, 3, 2 ]}\nobreak
\hyperdef{L}{X87C24CF085566CEA}{}
{\noindent\textcolor{FuncColor}{$\triangleright$\ \ \texttt{Genus({\mdseries\slshape E})\index{Genus@\texttt{Genus}!equation}
\label{Genus:equation}
}\hfill{\scriptsize (operation)}}\\
\textbf{\indent Returns:\ }
The integer that is the genus of the equation

}

 

\subsection{\textcolor{Chapter }{EquationSignature (equation)}}
\logpage{[ 4, 3, 3 ]}\nobreak
\hyperdef{L}{X801F396A7F871C54}{}
{\noindent\textcolor{FuncColor}{$\triangleright$\ \ \texttt{EquationSignature({\mdseries\slshape E})\index{EquationSignature@\texttt{EquationSignature}!equation}
\label{EquationSignature:equation}
}\hfill{\scriptsize (operation)}}\\
\textbf{\indent Returns:\ }
The list \mbox{\texttt{\mdseries\slshape [n,m]}} of integers that is the signature of the equation

}

 }

 }

 
\chapter{\textcolor{Chapter }{FR-Equations}}\logpage{[ 5, 0, 0 ]}
\hyperdef{L}{X7BA9260C7CF0AAE4}{}
{
 
\section{\textcolor{Chapter }{Decomposable equations}}\logpage{[ 5, 1, 0 ]}
\hyperdef{L}{X84C6E3E57E19EE72}{}
{
 For self-similar groups one strategy to solve equations is to consider the
inherit equations by passing to states. To use this methods the package FR (\cite{FR2.3.6}) from Laurent Bartholdi is needed. 

 Let $G$ be a group which lies in the filter \mbox{\texttt{\mdseries\slshape IsFRGroup}} and which admitts an embedding $\psi\colon G \to \tilde G\wr S_n$ where $\tilde G $ is the group generated by the states of the group $G$. Note that if $G$ is a \emph{self-similar} group then $G\simeq \tilde G$. Further let $F_X$ be the free group on the generating set $X$. Given an equation group $G*F_X$ we will the fix $n$ natural embeddings $\varphi_i\colon F\to F_{X^n}$ and call the group $(\tilde G*F_{X^n})\wr S_n )$ the \emph{decomposition equation group} of $G*F_X$. The decomposition of an equation $e$ with variables $x_1,\ldots ,x_k$ with respect to a choice of activities $\sigma(x_i)\in S_n$ for each variable $x_i$ is the image of $e$ under the homomorphism \begin{center}
\begin{tabular}{rl}$\Phi_\sigma\colon G*F_X$&
$\to (\tilde G*(F_{X^n})\wr S_n$\\
$x_i$&
$\mapsto\varphi_i(x_i)\cdot\sigma(x_i)$\\
$g$&
$\mapsto\psi_i(x_i)$\\
\end{tabular}\\[2mm]
\end{center}

 

\subsection{\textcolor{Chapter }{DecompositionEquationGroup (group)}}
\logpage{[ 5, 1, 1 ]}\nobreak
\hyperdef{L}{X8431319680D47763}{}
{\noindent\textcolor{FuncColor}{$\triangleright$\ \ \texttt{DecompositionEquationGroup({\mdseries\slshape G})\index{DecompositionEquationGroup@\texttt{DecompositionEquationGroup}!group}
\label{DecompositionEquationGroup:group}
}\hfill{\scriptsize (operation)}}\\
\textbf{\indent Returns:\ }
A new \mbox{\texttt{\mdseries\slshape EquationGroup}}.



 This method needs \mbox{\texttt{\mdseries\slshape G}} to be an equation group where the group of constants is an fr-group. For \mbox{\texttt{\mdseries\slshape G}} a group with free constant group see \texttt{DecompositionEquationGroup} (\ref{DecompositionEquationGroup:group,int,list}). If $F$ is the free group on the generating set $X$ then the free group on the gerating set $X^n$ is isomorphic to $F^{*n}$ the $n$-fold free product of $F$ . 

 This method returns the \mbox{\texttt{\mdseries\slshape EquationGroup}} $G*F^{*n}$.

 \noindent\textcolor{FuncColor}{$\triangleright$\ \ \texttt{DecompositionEquationGroup({\mdseries\slshape G, deg, acts})\index{DecompositionEquationGroup@\texttt{DecompositionEquationGroup}!group,int,list}
\label{DecompositionEquationGroup:group,int,list}
}\hfill{\scriptsize (operation)}}\\
\textbf{\indent Returns:\ }
A new \mbox{\texttt{\mdseries\slshape EquationGroup}}.



 This method needs \mbox{\texttt{\mdseries\slshape G}} to be an equation group where the group of constants is a free group on $n<\infty$ generators. The integer \mbox{\texttt{\mdseries\slshape deg}} is the number of states each element will have. The list \mbox{\texttt{\mdseries\slshape acts}} should be of length $n$ and all elements should be permutation of \mbox{\texttt{\mdseries\slshape deg}} elements. These will represent the activity of the generators of the free
group. }

 

\subsection{\textcolor{Chapter }{DecompositionEquation (equation)}}
\logpage{[ 5, 1, 2 ]}\nobreak
\hyperdef{L}{X7FA2A56D79F97366}{}
{\noindent\textcolor{FuncColor}{$\triangleright$\ \ \texttt{DecompositionEquation({\mdseries\slshape G, E, sigma})\index{DecompositionEquation@\texttt{DecompositionEquation}!equation}
\label{DecompositionEquation:equation}
}\hfill{\scriptsize (operation)}}\\
\textbf{\indent Returns:\ }
A new equation in \mbox{\texttt{\mdseries\slshape G}} which is the decomposed of the equation \mbox{\texttt{\mdseries\slshape E}}.



 The group \mbox{\texttt{\mdseries\slshape G}} needs to be a \mbox{\texttt{\mdseries\slshape DecompositionEqationGroup(H)}}, the equation \mbox{\texttt{\mdseries\slshape E}} needs to be a member of the EquationGroup $H=K*F$. 

 The argument \mbox{\texttt{\mdseries\slshape sigma}} needs to be a group homomorphism $\sigma\colon F\to S_n$. Alternatively it can be a list of elements of $S_n$ it is then regarded as the group homomorphism that maps the $i$-th variable of \mbox{\texttt{\mdseries\slshape eq}} to the $i$-th element of the list. 

 The representation of the returned equation stores a list of words such that
the $i$-th word represents an element in $G*\phi_i(F)$. 
\begin{Verbatim}[commandchars=!@|,fontsize=\small,frame=single,label=Example]
  !gapprompt@gap>| !gapinput@F := FreeGroup(1);; SetName(F,"F");|
  !gapprompt@gap>| !gapinput@G := EquationGroup(GrigorchukGroup,F);|
  GrigorchukGroup*F
  !gapprompt@gap>| !gapinput@DG := DecompositionEquationGroup(G);|
  GrigorchukGroup*F*F
  !gapprompt@gap>| !gapinput@ sigma := GroupHomomorphismByImages(F,SymmetricGroup(2),[(1,2)]);|
  [ f1 ] -> [ (1,2) ]
  !gapprompt@gap>| !gapinput@ e := Equation(G,[F.1^2,GrigorchukGroup.2]);|
  Equation in [ f1 ]
  !gapprompt@gap>| !gapinput@ de := DecompositionEquation(DG,e,sigma);|
  DecomposedEquation in [ f11, f12 ]
  !gapprompt@gap>| !gapinput@Print(de);|
  Equation([ FreeProductElm([ f11*f12,a ]), FreeProductElm([ f12*f11,c ]) ])
\end{Verbatim}
 
\begin{Verbatim}[commandchars=!@|,fontsize=\small,frame=single,label=Example]
  !gapprompt@gap>| !gapinput@F := FreeGroup("x1","x2");; SetName(F,"F");|
  !gapprompt@gap>| !gapinput@G := FreeGroup("g");; SetName(G,"G");|
  !gapprompt@gap>| !gapinput@eG := EquationGroup(G,F);|
  G*F
  !gapprompt@gap>| !gapinput@DeG := DecompositionEquationGroup(eG,2,[(1,2)]);|
  G*G*F*F
  !gapprompt@gap>| !gapinput@ e := Equation(eG,[Comm(F.1,F.2),G.1^2]);|
  Equation in [ x1, x2 ]
  !gapprompt@gap>| !gapinput@ Print(DecompositionEquation(DeG,e,[(),()]));|
  Equation([ FreeProductElm([ x11^-1*x21^-1*x11*x21, g1*g2 ]), 
   FreeProductElm([ x12^-1*x22^-1*x12*x22, g2*g1 ]) ])
\end{Verbatim}
 }

 

\subsection{\textcolor{Chapter }{EquationComponent (equation,int)}}
\logpage{[ 5, 1, 3 ]}\nobreak
\hyperdef{L}{X7EA966847C9791A0}{}
{\noindent\textcolor{FuncColor}{$\triangleright$\ \ \texttt{EquationComponent({\mdseries\slshape E, i})\index{EquationComponent@\texttt{EquationComponent}!equation,int}
\label{EquationComponent:equation,int}
}\hfill{\scriptsize (operation)}}\\
\textbf{\indent Returns:\ }
The \mbox{\texttt{\mdseries\slshape i}}-th component of the decomposed equation \mbox{\texttt{\mdseries\slshape E}}.



 Denote by $p_i$ the natural projection $(G*F_{X^n})^n\rtimes S_n\to G*F_{X^n}$ to the $i$-th factor of the product. Given a decomposed Equation \mbox{\texttt{\mdseries\slshape E}} and an integer $0<$\mbox{\texttt{\mdseries\slshape i}}$\leq n$ this method returns $p_i(E)$.

 \noindent\textcolor{FuncColor}{$\triangleright$\ \ \texttt{EquationComponents({\mdseries\slshape E})\index{EquationComponents@\texttt{EquationComponents}!equation,int}
\label{EquationComponents:equation,int}
}\hfill{\scriptsize (operation)}}\\
\textbf{\indent Returns:\ }
The list of all components of the decomposed equation \mbox{\texttt{\mdseries\slshape E}}.



 Denote by $p_i$ the natural projection $p_i\colon(G*F_{X^n})^n\rtimes S_n\to G*F_{X^n}$ to the $i$-th factor of the product. Given a decomposed Equation \mbox{\texttt{\mdseries\slshape E}} this method returns the list $[p_1(E),p_2(E),\ldots,p_n(E)]$.

 \noindent\textcolor{FuncColor}{$\triangleright$\ \ \texttt{EquationActivity({\mdseries\slshape E})\index{EquationActivity@\texttt{EquationActivity}!equation}
\label{EquationActivity:equation}
}\hfill{\scriptsize (operation)}}\\
\textbf{\indent Returns:\ }
The activity of the decomposed equation \mbox{\texttt{\mdseries\slshape E}}.



 Denote by $act$ the natural projection $(G*F_{X^n})\wr S_n\to S_n$. Given a decomposed Equation \mbox{\texttt{\mdseries\slshape E}} this method returns $act(E)$. }

 

\subsection{\textcolor{Chapter }{DecomposedEquationDisjointForm (equation)}}
\logpage{[ 5, 1, 4 ]}\nobreak
\hyperdef{L}{X83F46D707CDA0EEE}{}
{\noindent\textcolor{FuncColor}{$\triangleright$\ \ \texttt{DecomposedEquationDisjointForm({\mdseries\slshape E})\index{DecomposedEquationDisjointForm@\texttt{DecomposedEquationDisjointForm}!equation}
\label{DecomposedEquationDisjointForm:equation}
}\hfill{\scriptsize (operation)}}\\
\textbf{\indent Returns:\ }
A record with components \mbox{\texttt{\mdseries\slshape eq}} and \mbox{\texttt{\mdseries\slshape hom}}.



 If \mbox{\texttt{\mdseries\slshape E}} is a decomposed equation there may be an overlap of the set of variables of
some components. If \mbox{\texttt{\mdseries\slshape E}} is a quadratic equation there is an equation homomorphism $\varphi$ that maps each component to a new quadratic equation. Hence all maped
components have pairwise disjoint sets of variables. This method computes such
an homomorphism $\varphi$ such that the solvability of the system of components remains unchanged. If $s$ is a solution for the new system of components, then $s\circ\varphi$ is a solution for the old system.

 The method returns a record with two components. \mbox{\texttt{\mdseries\slshape hom}} is the homomorphism $\varphi$ and \mbox{\texttt{\mdseries\slshape eq}} the new decomposed equation. }

 

\subsection{\textcolor{Chapter }{LiftSolution (equation,equation,equationhom,equationhom)}}
\logpage{[ 5, 1, 5 ]}\nobreak
\hyperdef{L}{X7F70CE8C780DE3C2}{}
{\noindent\textcolor{FuncColor}{$\triangleright$\ \ \texttt{LiftSolution({\mdseries\slshape DE, E, sigma, sol})\index{LiftSolution@\texttt{LiftSolution}!equation,equation,equationhom,equationhom}
\label{LiftSolution:equation,equation,equationhom,equationhom}
}\hfill{\scriptsize (operation)}}\\
\textbf{\indent Returns:\ }
An evaluation for E \mbox{\texttt{\mdseries\slshape eq}}.



 Given an equation \mbox{\texttt{\mdseries\slshape E}} and a solution \mbox{\texttt{\mdseries\slshape sol}} for its decomposed equation \mbox{\texttt{\mdseries\slshape DE}} under the decomposition with activity \mbox{\texttt{\mdseries\slshape sigma}} this method computes a solution for the equation \mbox{\texttt{\mdseries\slshape E}}.

 Note that the solution not neccecarily maps to the group of constants of \mbox{\texttt{\mdseries\slshape E}} but can map to the group where all elements of the group of constants can
appear as states. If the group of constants is layered, this two groups will
coincide. 
\begin{Verbatim}[commandchars=!@|,fontsize=\small,frame=single,label=Example]
  !gapprompt@gap>| !gapinput@F := FreeGroup(2);; SetName(F,"F");|
  !gapprompt@gap>| !gapinput@Gr := GrigorchukGroup;; a:=Gr.1;; d:=Gr.4;;|
  !gapprompt@gap>| !gapinput@G := EquationGroup(Gr,F);;|
  !gapprompt@gap>| !gapinput@DG := DecompositionEquationGroup(G);;|
  !gapprompt@gap>| !gapinput@ sigma := GroupHomomorphismByImages(F,SymmetricGroup(2),[(1,2),()]);|
  [ f1, f2 ] -> [ (1,2), () ]
  !gapprompt@gap>| !gapinput@ e := Equation(G,[Comm(F.1,F.2),Comm(d,a)]);|
  Equation in [ f1, f2 ]
  !gapprompt@gap>| !gapinput@ de := DecompositionEquation(DG,e,sigma);|
  DecomposedEquation in [ f11, f21, f12, f22 ]
  !gapprompt@gap>| !gapinput@dedj := DecomposedEquationDisjointForm(de);|
  rec( eq := DecomposedEquation in [ f11, f12, f22 ], 
    hom := [ f21 ]"->"[ FreeProductElm of length 3 ] )
  !gapprompt@gap>| !gapinput@EquationComponents(dedj.eq);|
  [ Equation in [ f11, f12, f22 ], Equation in [  ] ]
  !gapprompt@gap>| !gapinput@s := EquationEvaluation(DG,EquationVariables(dedj.eq),[One(Gr),One(Gr),Gr.2]);|
  MappingByFunction( GrigorchukGroup*F*F, GrigorchukGroup, function( q ) ... end )
  !gapprompt@gap>| !gapinput@IsSolution(s,EquationComponent(dedj.eq,1));|
  true
  !gapprompt@gap>| !gapinput@ns := dedj.hom*s;; IsEvaluation(ns);|
  true
  !gapprompt@gap>| !gapinput@ForAll(EquationComponents(de),F->IsSolution(ns,F));|
  true
  !gapprompt@gap>| !gapinput@ls := LiftSolution(de,e,sigma,ns);;|
  !gapprompt@gap>| !gapinput@IsSolution(ls,e);|
  true
  !gapprompt@gap>| !gapinput@ForAll(EquationVariables(e),x->Equation(G,[x])^ls in Gr);|
  true //only good luck
\end{Verbatim}
 }

 }

 }

 \def\bibname{References\logpage{[ "Bib", 0, 0 ]}
\hyperdef{L}{X7A6F98FD85F02BFE}{}
}

\bibliographystyle{alpha}
\bibliography{bio.xml}

\addcontentsline{toc}{chapter}{References}

\def\indexname{Index\logpage{[ "Ind", 0, 0 ]}
\hyperdef{L}{X83A0356F839C696F}{}
}

\cleardoublepage
\phantomsection
\addcontentsline{toc}{chapter}{Index}


\printindex

\newpage
\immediate\write\pagenrlog{["End"], \arabic{page}];}
\immediate\closeout\pagenrlog
\end{document}
